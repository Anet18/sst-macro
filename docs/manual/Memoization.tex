% !TEX root = manual.tex

\section{Memoization pragmas}\label{sec:memoization}

\subsection{Memoization models}
To understand the memoization pragmas, we first introduce how models get constructed in the SST/macro runtime.
Source-to-source transformations based on the pragmas causes the following hooks to get inserted:

\begin{CppCode}
int sstmac_start_memoize(const char* token, const char* model);
void sstmac_finish_memoize0(int tag, const char* token);
void sstmac_finish_memoize1(int tag, const char* token, double p1);
void sstmac_finish_memoize2(int tag, const char* token, double p1, double p2);
...
\end{CppCode}
A start call begins a memoization region for a specific name.
The start function must return an integer tag identifying the memoization instance.
This tag gets passed back into the finish function above.
This is primarily useful for thread-safe collection, but can be generally more useful.
The finish functions take input parameters. 
Given input parameters $x$,$y$ causes a function $F(x,y)$ to be fit to the timer or performance counters.

If building a skeleton application that uses memoization data, a different hook gets inserted:
\begin{CppCode}
void sstmac_compute_memoize0(const char* token);
void sstmac_compute_memoize1(const char* token, double p1);
void sstmac_compute_memoize2(const char* token, double p1, double p2);
...
\end{CppCode}
Assuming a model $F(x,y)$ has been fit in a memoization pass,
that function is invoked with the given parameters to estimate a time or performance counter.

Memoization models are implemented by inheriting from a standard class

\begin{CppCode}
struct RegressionModel {
...
virtual double compute(int n_params, const double params[], ImplicitState* state) = 0;
virtual int StartCollection() = 0;
virtual void finishCollection(int n_params, const double params[], ImplicitState* state) = 0;
...
\end{CppCode}
A call to \inlinecode{sstmac_finish_memoize2} causes \inlinecode{finishCollection(2,..)} to get invoked on the model.
The \inlinecode{states} object is discussed more later in \ref{subsec:implicitStates}.
For now, \inlinecode{compute} only returns a double (total time).
Generalized performance models are planned for future versions.
Models are registered using the SST/macro factory system. 
If wanting to add a least-squares model, factory register as:

\begin{CppCode}
struct least_squares : public regression model {
 FactoryRegister("least_squares", OperatingSystem::RegressionModel, least_squares)
\end{CppCode}

\subsection{pragma sst memoize [skeletonize(...)] [model(...)] [inputs(...)] [name(...)]}
\begin{itemize}
\item skeletonize: boolean for whether code block should still be executed or remove entirely (default: true)
\item model: string name for a type of model (e.g. linear, kmeans) specifying which model to construct and fit (no default)
\item inputs: a comma-separated list of C++ expressions that are the numeric inputs
\item name: a unique name to use for identifying the memoization region (default: see below)
\end{itemize}
If the \inlinecode{name} parameter is not given, the file and line number is used for basic expressions while the function name is used if applied to a function.
Consider the example:

\begin{CppCode}
#pragma sst memoize skeletonize(true) model(least_squares) inputs(ncol,nlink,nrow) 
void dgemm(int ncol, int nlink, int nrow, double* left, double* right);
\end{CppCode}
When running the memoization pass, the memoization hooks get invoked as:

\begin{CppCode}
int tag = sstmac_start_memoize("dgemm", "least_squares");
dgemm(....);
sstmac_finish_memoize3(tag, "dgemm", ncol, nlink, nrow);
\end{CppCode}
With \inlinecode{skeletonize} set to true, the skeleton app would be:

\begin{CppCode}
sstmac_computeMemoize("dgemm", ncol, nlink, nrow);
\end{CppCode}
With skeletonize set to false:

\begin{CppCode}
sstmac_computeMemoize("dgemm", ncol, nlink, nrow);
dgemm(...);
\end{CppCode}
Both the memoization function and the original function would both get invoked.

\subsection{pragma sst implicit\_state X(Y) ...}\label{subsec:implicitStates}
The implicit state pragma sets certain hardware or software states not captured by the inputs to the memoization pragma.
This might involve DVFS states, different runs of a task in which data is ``cold'' or ``hot'' in cache, or different types of cores.
The implicit state lasts for the scope of the statement:

\begin{CppCode}
#pragma sst implict_state ...
{
 //all statements here have that state
}

#pragma sst ImplicitState ...
fxn(...) //implicit state lasts the entire function
\end{CppCode}

The arguments to the pragma are best understood by example:

\begin{CppCode}
#pragma sst ImplicitState dvfs(1) cache(hot)
fxn(...)
\end{CppCode}
This causes a source code transformation to:

\begin{CppCode}
sstmac_set_ImplicitState2(dvfs,1,cache,hot);
fxn(...);
sstmac_unset_ImplicitState2(dvfs,cache);
\end{CppCode}
For now, the functions take integer arguments (this may get relaxed to arbitrary strings).
Thus, e.g. enums must be available or compilation will fail:

\begin{CppCode}
enum states {
 dvfs=0,
 cache=1
};
enum cache_states {
 cold=0,
 hot=1
};
\end{CppCode}

If a \inlinecode{sstmac_finish_memoize} function got invoked, the states could be read.
The class \inlinecode{ImplicitState} is a base class only and carries no data by default.
Specific memoization models are intended to be used only with known implicit state classes.
As such, the memoization model \inlinecode{collect}, etc, functions must dynamic cast to an expected type.
A library of standard implicit state implementations is planned for future releases.
