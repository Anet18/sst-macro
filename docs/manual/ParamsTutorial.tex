%% !TEX root = manual.tex

\section{SST/macro Parameter files}
\label{sec:parameters}

There are parameter files for the main network models (MACRELS, PISCES, SCULPIN, SNAPPR) in the top-level examples directory.
A minimal parameter file setting up a 2D-torus topology is shown below. 
An equivalent Python input file that reads an ini file is also shown.
A detailed listing of parameter namespaces and keywords is given in Section \ref{chapter:parameters}.
Both the \inlineshell{ini} files and Python files make careful use of namespaces.

\begin{ViFile}
amm_model = amm1
congestion_model = LogP
node {
 #run a single mpi test
 app1 {
  indexing = block
  allocation = first_available
  launch_cmd = aprun -n8 -N1
  name = sstmac_mpi_testall
  argv =
  sendrecvMessage_size = 128
 }
 ncores = 1
 memory {
  model = simple
  bandwidth = 1GB/s
  latency = 10ns
 }
 proc {
  frequency = 1GHz
 }
 nic {
  injection {
   bandwidth = 1GB/s
   latency = 1us
  }
  model = simple
 }
}
switch {
 link {
  bandwidth = 1.0GB/s
   latency = 100ns
 }
 logp {
   bandwidth = 1GB/s
   out_in_latency = 1us
 }
}

topology {
 name = torus
 geometry = 4,4
}
\end{ViFile}
The input file follows a basic syntax of \inlinefile{parameter = value}.
Parameter names follow C++ variable rules (letters, numbers, underscore) while parameter values can contain spaces.  Trailing and leading whitespaces are stripped from parameters.
Comments can be included on lines starting with \#.

\subsection{Parameter Namespace Rules}
\label{subsec:parameterNamespace}
Periods denote nesting of parameter namespaces.
The parameter \inlineshell{node.memory.model} will be nested in namespace \inlineshell{memory} inside namespace \inlineshell{node}.
If inside a given namespace, \sstmacro looks only within that namespace.

The preferred syntax more closely resembles C++ namespace declarations. 
Namespaces are scoped using brackets \{\}:

\begin{ViFile}
node {
 model = simple
 memory {
   model = simple
   bandwidth = 1GB/s
   latency = 10ns
 }
}
\end{ViFile}
Any line containing a single string with an opening \{ starts a new namespace.
A line containing only a closing \} ends the innermost namespace.
The syntax is not as flexible as C++ since the opening \{ must appear on the same line as the namespace and the closing \} must be on a line of its own.
A detailed listing of parameter namespaces and keywords is given in Section \ref{chapter:parameters}.

\subsection{Initial Example}
\label{subsec:initialExample}
Continuing with the example above, we see the input file is broken into namespace sections. 
First, application launch parameters for each node must be chosen determining what application will launch, 
how nodes will be allocated, how ranks will be indexed, and finally what application will be run.
Additionally, you must specify how many processes to launch and how many to spawn per node.
We currently recommend using aprun syntax (the launcher for Cray machines),
although support is being added for other process management systems.
\sstmacro can simulate command line parameters by giving a value for \inlinefile{node.app1.argv}.

A network must also be chosen.
In the simplest possible case, the network is modeled via a simple latency/bandwidth formula.
For more complicated network models, many more than two parameters will be required.
See \ref{sec:tutorial:networkmodel} for a brief explanation of \sstmacro network congestion models.
A topology is also needed for constructing the network.
In this case we choose a 2-D 4$\times$4 torus (16 switches).  The \inlinefile{topology.geometry}
parameter takes an arbitrarily long list of numbers as the dimensions to the torus.

Finally, we must construct a node model.
In this case, again, we use the simplest possible models for the node,
network interface controller (NIC), and memory.

Parameter files can be constructed in a more modular way through the \inlinefile{include} statement.
An alternative parameter file would be:

\begin{ViFile}
include machine.ini
# Launch parameters
node {
 app1 {
  indexing = block
  allocation = first_available
  launch_cmd = aprun -n2 -N1
  name = user_mpiapp_cxx
  argv = 
  # Application parameters
  sendrecvMessage_size = 128
 }
}

\end{ViFile}
where in the first line we include the file \inlinefile{machine.ini}.
All network, topology, and node parameters would be placed into a \inlinefile{machine.ini} file.
In this way, multiple experiments can be linked to a common machine.
Alternatively, multiple machines could be linked to the same application by creating and including an \inlinefile{application.ini}.
