%% !TEX root = manual.tex

\section{SST/macro Parameter files}
\label{sec:parameters}
A minimal parameter file setting up a 2D-torus topology is shown below. 
The preferred (current) version uses namespaces (ns.param syntax).
However, we also show the deprecated parameters used by previous versions.

\begin{ViFile}
# Launch parameters
app1.launch_indexing = block
app1.launch_allocation = first_available
app1.launch_cmd = aprun -n2 -N1
app1.name = user_app_cxx
app1.argv = 
# Application parameters
app1.sendrecv_message_size = 128
# Network parameters
interconnect = simple
switch.bandwidth = 1.0GB/s
switch.hop_latency = 100ns
# Topology - Ring of 4 nodes
topology.name = hdtorus
topology.geometry = 4,4
# Node parameters
node.ncores = 1
node.model = null
node.memory.model = null
nic.model = null
\end{ViFile}
The input file follows a basic syntax of \inlinefile{parameter = value}.  
Parameter names follow C++ variable rules (letters, numbers, underscore) while parameter values can contain spaces.  Trailing and leading whitespaces are stripped from parameters.
Comments can be included on lines starting with \#.

The input file is broken into sections via comments.  
First, application launch parameters must be chosen determining what application will launch, 
how nodes will be allocated, how ranks will be indexed, and finally what application will be run.  
Additionally, you must specify how many processes to launch and how many to spawn per node.  
We currently recommend using aprun syntax (the launcher for Cray machines), 
although support is being added for other process management systems.
\sstmacro can simulate command line parameters by giving a value for \inlinefile{app1.argv}.

A network must also be chosen.  
In the simplest possible case, the network is modeled via a simple latency/bandwidth formula.  
For more complicated network models, many more than two parameters will be required. 
See \ref{sec:tutorial:networkmodel} for a brief explanation of \sstmacro network congestion models. 
A topology is also needed for constructing the network.  
In this case we choose a 2-D 4$\times$4 torus (16 switches).  The \inlinefile{topology_geometry} 
parameter takes an arbitrarily long list of numbers as the dimensions to the torus.

Finally, we must construct a node model.  
In this case, again, we use the simplest possible models (null model) for the node, 
network interface controller (NIC), and memory.  
The null model is essentially a no-op, generating the correct control flow but not actually simulating any computation. 
This is useful for validating program correctness or examining questions only related to the network.  
More accurate (and complicated) models will require parameters for node frequency, memory bandwidth, injection latency, etc.

Parameter files can be constructed in a more modular way through the \inlinefile{include} statement.  
An alternative parameter file would be:

\begin{ViFile}
include machine.ini
# Launch parameters
app1.launch_indexing = block
app1.launch_allocation = first_available
app1.launch_cmd = aprun -n2 -N1
app1.name = user_mpiapp_cxx
app1.argv = 
# Application parameters
app1.sendrecv_message_size = 128
\end{ViFile}
where in the first line we include the file \inlinefile{machine.ini}.  
All network, topology, and node parameters would be placed into a \inlinefile{machine.ini} file.  
In this way, multiple experiments can be linked to a common machine.  
Alternatively, multiple machines could be linked to the same application by creating and including an \inlinefile{application.ini}.

Using the deprecated (non-namespace) parameters the file would be:

\begin{ViFile}
# Launch parameters
launch_indexing = block
launch_allocation = first_available
launch_app1_cmd = aprun -n2 -N1
launch_app1 = user_mpiapp_cxx
launch_app1_argv = 
# Network parameters
network_name = simple
network_bandwidth = 1.0GB/s
network_hop_latency = 100ns
# Topology - Ring of 4 nodes
topology_name = hdtorus
topology_geometry = 4,4
# Node parameters
node_cores = 1
node_name = null
node_memory_model = null
nic_name = null
# Application parameters
sendrecv_message_size = 128
\end{ViFile}

