%% !TEX root = manual.tex

\section{SST/macro Parameter files}
\label{sec:parameters}
A minimal parameter file setting up a 2D-torus topology is shown below. 
A detailed listing of parameter namespaces and keywords is given in Section \ref{chapter:parameters}.
The preferred input files now use namespaces.
However, for consistency with previous versions, we also show the deprecated parameters.

\begin{ViFile}
# Launch parameters
app1.indexing = block
app1.allocation = first_available
app1.launch_cmd = aprun -n8 -N1
app1.name = sstmac_mpi_testall
app1.argv =
# Application parameters
app1.sendrecv_message_size = 128
# Network parameters
amm_model = amm1
congestion_model = simple
switch.ejection.bandwidth = 1GB/s
switch.ejection.latency = 100ns
switch.link.bandwidth = 1.0GB/s
switch.link.latency = 100ns
# Topology - 4x4 Torus
topology.name = torus
topology.geometry = 4,4
# Node parameters
node.ncores = 1
node.model = simple
node.memory.model = simple
node.memory.bandwidth = 1GB/s
node.memory.latency = 10ns
node.proc.frequency = 1GHz
nic.model = simple
nic.injection.bandwidth = 1GB/s
nic.injection.latency = 1us
\end{ViFile}
The input file follows a basic syntax of \inlinefile{parameter = value}.  
Parameter names follow C++ variable rules (letters, numbers, underscore) while parameter values can contain spaces.  Trailing and leading whitespaces are stripped from parameters.
Comments can be included on lines starting with \#.

\subsection{Parameter Namespace Rules}
\label{subsec:parameterNamespace}
Periods denote nesting of parameter namespaces.
The parameter \inlineshell{node.memory.model} will be nested in namespace \inlineshell{memory} inside namespace \inlineshell{node}.
SST/macro generally requests variable values via the inner-most namespace only.
This means rather than asking for variable \inlineshell{node::memory::model} it will actually look for variable \inlineshell{memory::model} starting with the most deeply nested namespace.
This exactly follows C++ namespace rules.

For example, consider the following:

\begin{ViFile}
memory.model = simple
node.memory.model = pisces
\end{ViFile}
If I am building the node's memory system, the initialization will look for \inlineshell{memory::model} inside namespace \inlineshell{node} first, returning the value \inlineshell{pisces}.
If node initialization had been unable to find the variable in the \inlineshell{node} namespace,
it would have moved up and looked in the global namespace, returning the value \inlineshell{simple}.

A new syntax supported in 6.1 more closely resembles C++ namespace declarations. 
Namespaces can be scoped using brackets \{\}:

\begin{ViFile}
node {
 model = simple
 memory {
   model = simple
   bandwidth = 1GB/s
   latency = 10ns
 }
}
\end{ViFile}
Any line containing a single string with an opening \{ starts a new namespace.
A line containing only a closing \} ends the innermost namespace.
The syntax is not as flexible as C++ since the opening \{ must appear on the same line as the namespace and the closing \} must be on a line of its own.

Using the new syntax, the parameter file above now becomes:

\begin{ViFile}
amm_model = amm1
congestion_model = simple

app1 {
 indexing = block
 allocation = first_available
 launch_cmd = aprun -n8 -N1
 name = sstmac_mpi_testall
 argv =
 sendrecv_message_size = 128
}

switch {
 ejection {
  bandwidth = 1GB/s
  latency = 100ns
 }
 link {
  bandwidth = 1.0GB/s
  latency = 100ns
 }
}

topology {
 name = torus
 geometry = 4,4
}

node {
 ncores = 1
 model = simple
 memory {
  bandwidth = 1GB/s
  latency = 10ns
  model = simple
 }
 proc.frequency = 1GHz
 nic {
  injection.bandwidth = 1GB/s
  injection.latency = 1us
 }
}
\end{ViFile}
Again, a detailed listing of parameter namespaces and keywords is given in Section \ref{chapter:parameters}.

\subsection{Initial Example}
\label{subsec:initialExample}
Continuing with the example above, we see the input file is broken into sections via comments.  
First, application launch parameters must be chosen determining what application will launch, 
how nodes will be allocated, how ranks will be indexed, and finally what application will be run.  
Additionally, you must specify how many processes to launch and how many to spawn per node.  
We currently recommend using aprun syntax (the launcher for Cray machines), 
although support is being added for other process management systems.
\sstmacro can simulate command line parameters by giving a value for \inlinefile{app1.argv}.

A network must also be chosen.  
In the simplest possible case, the network is modeled via a simple latency/bandwidth formula.  
For more complicated network models, many more than two parameters will be required. 
See \ref{sec:tutorial:networkmodel} for a brief explanation of \sstmacro network congestion models. 
A topology is also needed for constructing the network.  
In this case we choose a 2-D 4$\times$4 torus (16 switches).  The \inlinefile{topology_geometry} 
parameter takes an arbitrarily long list of numbers as the dimensions to the torus.

Finally, we must construct a node model.  
In this case, again, we use the simplest possible models (null model) for the node, 
network interface controller (NIC), and memory.  
The null model is essentially a no-op, generating the correct control flow but not actually simulating any computation. 
This is useful for validating program correctness or examining questions only related to the network.  
More accurate (and complicated) models will require parameters for node frequency, memory bandwidth, injection latency, etc.

Parameter files can be constructed in a more modular way through the \inlinefile{include} statement.  
An alternative parameter file would be:

\begin{ViFile}
include machine.ini
# Launch parameters
app1.indexing = block
app1.allocation = first_available
app1.launch_cmd = aprun -n2 -N1
app1.name = user_mpiapp_cxx
app1.argv = 
# Application parameters
app1.sendrecv_message_size = 128
\end{ViFile}
where in the first line we include the file \inlinefile{machine.ini}.  
All network, topology, and node parameters would be placed into a \inlinefile{machine.ini} file.  
In this way, multiple experiments can be linked to a common machine.  
Alternatively, multiple machines could be linked to the same application by creating and including an \inlinefile{application.ini}.

Using the deprecated (non-namespace) parameters the file would be:

\begin{ViFile}
# Launch parameters
launch_indexing = block
launch_allocation = first_available
launch_app1_cmd = aprun -n2 -N1
launch_app1 = user_mpiapp_cxx
launch_app1_argv = 
# Network parameters
network_name = simple
network_bandwidth = 1.0GB/s
network_hop_latency = 100ns
# Topology - Ring of 4 nodes
topology_name = torus
topology_geometry = 4,4
# Node parameters
node_cores = 1
node_name = null
node_memory_model = null
nic_name = null
# Application parameters
sendrecv_message_size = 128
\end{ViFile}

