% !TEX root = manual.tex

\chapter{Uncertainty Quantification Methods and Tools}
\label{chapter:uq}

\section{Overview}
\label{sec:uqoverview}
For uncertainty quantification (UQ) and validation studies the UQ Toolkit (UQTk) library is being used.
UQTk (\url{www.sandia.gov/UQToolkit}) is a lightweight C++ library, developed in Sandia National Laboratories, California, 
that primarily offers tools for surrogate model construction and uncertainty propagation with 
polynomial chaos expansions (PCE), as well as model calibration and validation.

UQTk Version 2.0 is released under the GNU Lesser General Public License (LGPL). 

\section{Download and compilation}
\label{sec:uqdownload}

A tar-ball with the source code, tutorials, examples and documentation can be 
downloaded from \url{www.sandia.gov/UQToolkit/uqtk\_download.html}

Before compilation, one should set a file \texttt{config/config.site} with computer-specific compilation options. 
Both C++ and Fortran compilers are required. The file \texttt{config/config.gnu} provides a good 
starting point as a sample configuration file.

Compilation is simply invoked by 
\begin{ShellCmd}
make all
\end{ShellCmd}


\section{Structure}
\label{sec:uqstruct}

Below we present parts of the UQTk structure that are relevant to UQ and validation studies on \sstmacro.

\begin{itemize}
\item \texttt{config}: configuration file location
\item \texttt{doc\_cpp}: doxygen documentation
\item \texttt{src\_cpp}: source code - all libraries reside here
\subitem \texttt{src\_cpp/lib}: location of all compiled libraries in C++ and Fortran
\subitem \texttt{src\_cpp/apps}: command line utilities for performing various UQ tasks (source codes)
\subitem \texttt{src\_cpp/bin}: command line utilities for performing various UQ tasks (executables)
\item \texttt{examples\_cpp}: example codes and scripts
\subitem \texttt{examples\_cpp/uq\_surr}: a set of scripts that automates the forward UQ task utilizing the command line \texttt{apps} above.
\end{itemize}

\section{Applications/Capabilities}
\label{sec:uqapp}

The two basic UQ tasks, enabled by UQTk, are \emph{forward UQ} and \emph{inverse UQ} as outlined in the next subsections.

\subsection{Forward UQ: uncertainty propagation and global sensitivity analysis}
\label{subsec:fuq}

The main technique we employ for forward UQ is the spectral Polynomial Chaos expansions (PCEs). 
A PCE for a given model allows for a) efficient uncertainty propagation, b) very fast global sensitivity analysis, and 
c) cheap surrogate model construction that can replace the original model in sampling-intensive studies such as calibration, 
optimization, or, generally, inverse UQ.

The PCE library consists of a class for PC objects allowing both intrusive and non-intrusive uncertainty propagation. 
For \sstmacro, only non-intrusive methods are feasible that deal with the \sstmacro model as a \emph{black-box} without the need to rewrite or reformulate it.

\subsubsection{Generic workflow:}

A set of scripts that illustrate the essential forward UQ tasks is located in  \texttt{examples\_cpp/uq\_surr}.
To enable running all scripts one should set an environment variable \texttt{UQTK\_SRC} that points to the location of UQTk.

\begin{ShellCmd}
setenv UQTK_SRC location/of/uqtk/in/your/computer
\end{ShellCmd}


A generic workflow consists of 4 steps:
\begin{enumerate}
\item Generate parameter samples to run the forward model at, for PC construction and validation, \texttt{gen\_sam.x}
\begin{ShellCmd}
gen_sam.x <domain_file> <sampling_type(Q/U)> <N_samples> <N_val>
\end{ShellCmd}
The list of arguments:
\subitem \emph{domain\_file}: A file with \bgmth d \endmth rows and 2 columns. \bgmth d \endmth is the total number of parameters being explored.
           The two columns are the lower and upper bound of the corresponding parameter.
\subitem \emph{sampling\_type}: Q (Quadrature) or U (uniformly random). Note that currently in this script set, only Q is supported for further PCE generation.
\subitem \emph{N\_samples}: Number of samples used for training, i.e. for building PCE.
\subitem \emph{N\_val}:  Number of random samples generated for PCE validation. This can be set to 0 to skip validation.

\item Run the black-box model, \texttt{model.x}
\begin{ShellCmd}
model.x
\end{ShellCmd}
This is a black-box model that takes no arguments. However, it expects two input files mparam.dat and minput.dat, and returns the function evaluations in the output file moutput.dat.
\subitem \emph{mparam.dat}: a single column $d\times$1 of the parameters of interest, where d is the number of parameters 
\subitem \emph{minput.dat}: controllable input in a matrix form, $N$$\times$$k$, where $k$ is the number of controllable parameters and $N$ is the number of values or model observables.
\subitem \emph{moutput.dat}: the model output in a column format $N$$\times$1.

This is the file that needs to be modified/provided by the user according to the model under study. Currently, a simple function $y=Ae^{Bx}+2Bx$ is implemented, where $A$ and $B$ are the parameters (mparam.dat) and $x$ is the single controllable input parameter (minput.dat). A user-created \texttt{model.x} should accept input files minput.dat and mparam.dat as described above, and it should produce an output file moutput.dat with the formats described above, 

\item Obtain PCE for the model, \texttt{uq\_surr.x}
\begin{ShellCmd}
uq_surr.x <domain_file> <sampling_type(Q/U)> <N_samples> <N_val> \
		<P_order> <moutput_surr_filename> <moutput_val_filename>
\end{ShellCmd}

The first four arguments coincide with those from \texttt{gen\_sam.x}, the rest of the arguments are:
\subitem \emph{P\_order}: the PCE surrogate order. Polynomial series is truncated according to the total order.
\subitem \emph{moutput\_surr\_filename}: model output file resulting from running \texttt{model.x} on training samples.
\subitem \emph{moutput\_val\_filename}: model output file resulting from running \texttt{model.x} on validation samples.

\item Postprocess, e.g. global sensitivity analysis, \texttt{pp\_sens.x} 
\begin{ShellCmd}
pp_sens.x
\end{ShellCmd}

The current implementation of the post processing script \texttt{pp\_sens.x} takes no arguments. It expects the presence of a PCE information file pccf\_all.dat, which is one of the outcomes of \texttt{uq\_surr.x}. The final global sensitivity results are saved in the \inlineshell{allsens.dat} file of dimensions $N$$\times$$d$, where each row corresponds to a single value for the controllable input (number of controllable inputs = N), and each column corresponds to the sensitivity index of a parameter (number of parameters = d).\\

\medskip

Note that one can run the model \texttt{model.x} in an \emph{online} regime by using the keyword "M" instead of filenames in
\begin{ShellCmd}
uq_surr.x <domain_file> <sampling_type(Q/U)> <N_samples> <N_val> <P_order> M M
\end{ShellCmd}
that effectively incorporates the steps 1-3 above.

\subsubsection{Simple Example:}

The script \texttt{example.x} incorporates the full workflow above for a test function $y=Ae^{Bx}+2Bx$. Try
\begin{ShellCmd}
example.x online
\end{ShellCmd}
which produces, in a newly created folder \texttt{test}, the sensitivity file \inlineshell{allsens.dat} with sensitivity indices with respect to $A$ and $B$ for all values of controllable input $x$.



\end{enumerate}


\subsubsection{How Forward UQ applies to \sstmacro}

In the workflow described above, \sstmacro would replace the black-box simple model given by the equation, and \sstmacro parameters of interest would replace $A$ and $B$.   Note that while
we could explore the entire \sstmacro parameter space using this workflow, the higher the dimensionality of the space the more time process takes, so we would also want to narrow it down to the most important parameters, usually by reasoning, inspection, or simple experimentation.  
Also note that running an \sstmacro simulation can be considerably more expensive than evaluating an
analytical model, which is why we typically construct a \textit{surrogate}, or an arbitrarily-complex polynomial that can approximate performance given by \sstmacro. 

While the above workflow must currently be manually applied to \sstmacro data, we are working on automating this process and coming up with tutorials explaining how it is done.

\subsection{Inverse UQ: parameter calibration and model validation}
\label{subsec:iuq}

UQTk relies on Bayesian inference methods for model parameter calibration. Model validation is a direct result of calibration 
postprocessing. Indeed, a model is considered validated if the calibrated model parameters and the associated uncertainties can explain/predict available data well. 
The library \texttt{uqtkmcmc} allows implementation of Bayesian calibration using Markov chain Monte Carlo (MCMC) methods. The MCMC technique essentially 
searches the parameter space and compares model results with available data. Note that each parameter sample invokes a model evaluation, and often MCMC requires 
many samples for properly estimating the uncertainties. In such cases, the full model, as a black-box, will be replaced by its surrogate, again as a black-box model, that is constructed by the forward UQ techniques described in Subsection~\ref{subsec:fuq}.
