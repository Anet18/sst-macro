% !TEX root = manual.tex
\section{Torus}
\label{subsec:tutorial:hypercube}

The torus is the simplest topology and fairly easy to understand.
We have already discussed basic indexing and allocation as well as routing.
More complicated allocation schemes with greater fine-grained control can be used such as the
coordinate allocation scheme (see hypercube below for examples) and the node ID allocation scheme (see fat tree below for examples).
More complicated Valiant and UGAL routing schemes are shown below for hypercube and Cascade,
but apply equally well to torus.

For torus we illustrate here the Cartesian allocation for generating regular Cartesian subsets.
For this, the input file would look like 

\begin{ViFile}
topology {
 name = torus
 geometry = 4 4 4
}
node {
 app1 {
  launch_cmd = aprun -n 8
  indexing = block
  allocation = cartesian
  cart_sizes = [2,2,2]
  cart_offsets = [0,0,0]
 }
}
\end{ViFile}

This allocates a 3D torus of size 4x4x4.
Suppose we want to allocate all 8 MPI ranks in a single octant?
We can place them all in a 2x2x2 3D sub-torus by specifying the size of the sublock 
(\inlineshell{cart_sizes}) and which octant (\inlineshell{cart_offsets}).
This applies equally well to higher dimensional analogs.
This is particularly useful for allocation on Blue Gene machines
which always maintain contiguous allocations on a subset of nodes.

This allocation is slightly more complicated if we have multiple nodes per switch.
Even though we have a 3D torus, 
we treat the geometry as a 4D coordinate space with the 4th dimension referring to nodes connected to the same switch, 
i.e. if two nodes have the 4D coordinates [1 2 3 0] and [1 2 3 1] they are both connected to the same switch.
Consider the example below:

\begin{ViFile}
topology {
 name = torus
 geometry = 4 4 4
 concentration = 2
}
app1 {
 launch_cmd = aprun -n 8
 indexing = block
 allocation = cartesian
 cart_sizes = [2,2,1,2]
 cart_offsets = [0,0,0,0]
}
\end{ViFile}

We allocate a set of switches across an XY plane (2 in X, 2 in Y, 1 in Z for a single plane).
The last entry in \inlineshell{cart_sizes} indicates that both nodes on each switch should be used.


