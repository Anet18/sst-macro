%% !TEX root = manual.tex

\section{SST Python Files}
\label{sec:pythonFiles}
For SST core, SST/macro provides a Python translator for the ini files to a Python input deck.
For those wishing to directly use Python inputs (or understand better SST core files),
an example Python file is included here to illustrate using Macro.
Additionally, you can look at \inlineshell{sstmac/sst_core/sstmacro.py} to see more details of building and linking SST components.

SST/macro provides a particular idiom for setting up systems.
Rather than directly instantiate components, \inlineshell{sstmacro.py} provides an interconnect object.
Parameter namespaces are created through nested dictionaries.
One all parameters are created in the dictionaries, all the components are build and run through the \inlinecode{ic.build()} command.

\begin{ViFile}
import sst
import sst.macro
from sst.macro import Interconnect

link_bw="12.5GB/s"
link_lat="100ns"
logpParams = {
  "bandwidth" : "12.5GB/s",
  "hop_latency" : "100ns",
  "out_in_latency" : "1.2us",
}

swParams = {
  "name" : "pisces",
  "mtu" : "4KB",
  "arbitrator" : "cut_through",
  "router" : {
    "name"     :     "torus_minimal",
  },
  "link" : {
    "bandwidth" : link_bw,
    "latency" : link_lat,
    "credits" : "128KB",
  },
  "xbar" : {
    "bandwidth" : "1000GB/s",
    "latency" : "10ns",
  },
  "logp" : logpParams,
}

mpiParams = {
 "max_vshort_msg_size" : 4096,
 "max_eager_msg_size" : 64000,
 "post_header_delay" : "0.35906660us",
 "post_rdma_delay" : "0.88178612us",
 "rdma_pin_latency" : "5.42639881us",
 "rdma_page_delay" : "50.50000000ns",
}

appParams = {
  "allocation" : "hostname",
  "indexing" : "hostname",
  "exe" : "halo3d-26",
  "argv" : "-pex 4 -pey 4 -pez 4 -nx 128 -ny 128 -nz 128 -sleep 0 -iterations 10",
  "launch_cmd" : "aprun -n 64 -N 1",
  "allocation" : "first_available",
  "indexing" : "block",
  "mpi" : mpiParams,
}

memParams = {
 "name" : "pisces",
 "latency" : "10ns",
 "total_bandwidth" : "100GB/s",
 "max_single_bandwidth" : "11.20GB/s",
}

nicParams = {
  "name" : "pisces",
  "injection" : {
    "mtu" : "4KB",
    "redundant" : 8,
    "bandwidth" : "13.04GB/s",
    "arbitrator" : "cut_through",
    "latency" : "0.6us",
    "credits" : "128KB",
  },
  "ejection" : {
    "latency" : link_lat,
  },  
}

nodeParams = {
  "memory" : memParams,
  "nic" : nicParams,
  "app1" : appParams,
  "name" : "simple",
  "proc" : {
    "frequency" : "2GHz",
    "ncores" : "4",
  }
}

topoParams = {
 "name" : "torus,
 "geometry" : "[4,4,4]",
}

params = {
  "node" : nodeParams,
  "switch" : swParams,
  "topology" : topoParams,
}

ic = Interconnect(params)
ic.build()
\end{ViFile}

Again, to see all the details of creating and linking components, refer to the \inlineshell{sstmac/sst_core/sstmacro.py} file.


