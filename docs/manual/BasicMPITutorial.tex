%% !TEX root = manual.tex

\section{Basic MPI Program}
\label{sec:tutorial:basicmpi}
Let us go back to the simple send/recv skeleton and actually look at the code.  
This code should be compiled with SST compiler wrappers installed in the \inlineshell{bin} folder.

\begin{CppCode}
#include <stdlib.h>
#include <stdio.h>
#include <mpi.h>

#define sstmac_app_name simple_test

int main(int argc, char **argv) 
{
  int message_size = 128;
  int me, nproc;
  int tag = 0;
  int dst = 1;
  int src = 0;
  MPI_Status stat;

  MPI_Init(&argc,&argv);
  MPI_Comm world = MPI_COMM_WORLD;
  MPI_Comm_rank(world,&me);
  MPI_Comm_size(world,&nproc);
\end{CppCode}
The starting point is creating a main routine for the application.
The simulator itself already provides a \inlinecode{main} routine.
The SST compiler automatically changes the function name to \inlinecode{user_skeleton_main},
which provides an entry point for the application to actually begin.
When \sstmacro launches, it will invoke this routine and pass in any command line arguments specified via the \inlinefile{app1.argv} parameter.  Upon entering the main routine, 
the code is now indistinguishable from regular MPI C++ code.  
In the parameter file to be used with the simulation, you must set

\begin{ViFile}
app1.name = simple_test
\end{ViFile}

The name associated to the application is given by the \inlinecode{sstmac_app_name} macro.
This macro must be defined to a unique string name in the source file containing \inlinecode{main}.
\sstmacro will automatically associate the given main routine with the string internally.
That application can then be selected in the input file with \inlineshell{app1.name}.

At the very top of the file, the \inlineshell{mpi.h} header is actually mapped by the SST compiler to an \sstmacro header file.
This header provides the MPI API and configures MPI function calls to link to \sstmacro instead of the real MPI library.  
The code now proceeds:

\begin{CppCode}
  if (nproc != 2) {
    fprintf(stderr, "sendrecv only runs with two processors\n");
      abort();
  }
  if (me == 0) {
    MPI_Send(NULL, message_size, MPI_INT, dst, tag, world);
    printf("rank %i sending a message\n", me);
  }
  else {
    MPI_Recv(NULL, message_size, MPI_INT, src, tag, world, &stat);
    printf("rank %i receiving a message\n", me);
  }
  MPI_Finalize();
  return 0;
}
\end{CppCode}
Here the code just checks the MPI rank and sends (rank 0) or receives (rank 1) a message.
