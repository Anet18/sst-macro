%% !TEX root = manual.tex

%\section{Configuring MPI}
%\label{sec:tutorial:mpi}
%
%The actual implementation of MPI collectives and even the underlying protocol of MPI send/recv operations is highly configurable.
%Here we introduce the MPI tuning parameters and what they mean for performance experiments.
%
%\subsection{MPI Implementation}
%\label{subsec:mpi:implementation}
%
%The MPI implementation itself is flexible to change depending on how you want to utilize the network hardware.
%The default parameter is \inlineshell{mpi_implementation = basic},
%which only uses basic eager and rendezvous protocols and does not use RDMA.
%The cutoff for eager to rendezvous can be tuned by specifying, e.g.
%
%\begin{ViFile}
%mpi_handshake_size = 8KB
%\end{ViFile}
%giving the cutoff as an arbitrary byte length.
%
%For more complicated protocols, we can specify \inlineshell{mpi_implementation = rdma},
%which now allows a fancier set of protocols that leverage RDMA.
%The parameters here are tuned, e.g.
%
%\begin{ViFile}
%smp_single_copy_size = 8KB
%max_eager_msg_size = 8KB
%max_vshort_msg_size = 1KB
%\end{ViFile}
%
%The details of these protocols are best explained in the presentation ``Tuning the MPI Runtime Environment and Best Practices for Message Passing on Hopper''
%at \url{https://www.nersc.gov/assets/NUG-Meetings/2012/HowardP-MPI-NUG2012.pdf}.
%The parameter \inlineshell{smp_single_copy_size} is for intranode message passing.
%For small messages below the cutoff, an eager protocol is used where the message is copied into an intermediate buffer.
%For larger messages wishing to have zero-copy, a rendezvous protocol is used and the message is directly copied from send to receive buffer.
%The parameter \inlineshell{max_eager_msg_size} controls the crossover from an eager protocol using send buffers to a rendezvous protocol with zero-copy.
%Both the eager and rendezvous protocol use RDMA, creating a bit of a latency overhead.
%For very small messages, a special pathway can be taken avoiding RDMA and just directly sending the small message into a pre-allocated mailbox on the receiver.
%The cutoff for very small messages is given by \inlineshell{max_vshort_msg_size}.
%
%\subsection{MPI Collectives}
%The implementation of individual collectives can vary widely depending on the platform, message size, or distribution.
%\sstmacro provides limited flexibility for tuning collectives, but infrastructure is being developed to allow very fine-grained tuning
%of which MPI collective algorithms are used.  Currently, only MPI\_Allgather and MPI\_Allreduce can have alternative implementations chosen.
%For MPI\_Allgather, in the input file, one can specify \inlineshell{mpi_allgather = ring} or \inlineshell{mpi_allgather = bruck}
%to use either a basic ring algorithm or to use the Br{\"u}ck algorithm.  In the same way, one can either specify
%\inlineshell{mpi_allgather = rabenseifner} or \inlineshell{mpi_allgather = wilke}.
%More details can be found in ``Optimization of Collective Communication Operations in MPICH'' by Rajeev Thakur and Rolf Rabenseifner.
%The Wilke algorithm is a variation binary blocks algorithm described therein.
%Future versions of \sstmacro are expected to include a more complete library of collective algorithms with the ability to tune which algorithms are selected for specific buffer sizes.
%
%\subsection{MPI Queue}
%The implementation of the progress engine in MPI can either be a service (asynchronous progress thread distinct from the application) or an integrated part of the application.
%Most current MPI implementations have integrated progress engines that only move forward inside MPI\_Wait or MPI\_Test calls.
%However, you may wish to experiment with what performance gains can be had with an asynchronous progress thread.
%For the parameter \inlineshell{mpi_queue_thread_type}, you can either specify \inlineshell{user}, integrated progress engine that is part of the user application,
%or \inlineshell{service}, progress engine on asynchronous thread.
%If using the call graph feature (Section \ref{sec:tutorials:callgraph}), far more detail is available if using the integrated engine with \inlineshell{user}.