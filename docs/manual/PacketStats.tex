
%% !TEX root = manual.tex

\section{Network Statistics}
\label{sec:tutorials:packetStats}

Here we describe a few of the network statistics that can be collected and the basic mechanism for activating them.
These statistics are usually collected on either the NIC, switch crossbar, or switch output buffers.
These are based on the XmitWait and XmitBytes performance counters from OmniPath,
and aim to provide similar statistics as those from production systems.

\subsection{XmitBytes}
\label{subsec:xmitbytes}
To active a message size histogram on the NIC or the switches to determine the data sent by individual packets, the parameter file should include, for example:

\begin{ViFile}
node {
 nic {
  injection {
   xmit_bytes {
     output = csv
     type = accumulator
     group = test
   }
  }
 } 
}
\end{ViFile}
In contrast the custom statistics above, this is a row-table statistic that can have different types and outputs.
The same stat can be activated in both the \inlinecode{node.nic.injection} and \inlinecode{switch} namespaces.

\subsection{XmitWait}
\label{subsec:xmitwait}
To estimate congestion, SST/macro provides an \inlinecode{xmit_wait} statistic.
This counts the total amount of time spent in stalls due to lack of credits.
The time is accumulated and reported in seconds.

\begin{ViFile}
node {
 nic {
  injection {
   xmit_wait {
     output = csv
     type = accumulator
     group = test
   }
  }
 }
}
\end{ViFile}
The same stat can be activated in both the \inlinecode{node.nic.injection} and \inlinecode{switch} namespaces.

\subsection{XmitFlows}
\label{subsec:xmitflows}
The previous statistics track the bytes sent by packets and as such are agnostic to the messages (or flows) sending them.
If interested in the flow-level statistics, it can be activated as:

\begin{ViFile}
node {
 app1 {
  mpi {
    xmit_flows {

    }
  }
 }
}
\end{ViFile}


