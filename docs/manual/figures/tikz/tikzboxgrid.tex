
\newlength{\meshLength}
\newlength{\meshBoxWidth}
\newlength{\meshBoxHeight}
\newlength{\meshXSize}
\newlength{\meshYSize}
\newcommand{\boxGrid}[8]{
%#1 number in x dim
%#2 number in y dim
%#3 grid size
%#4 box width
%#5 box height
%#6 middle of mesh
%#7 box attributes
%#8 box name
\pgfmathtruncatemacro{\meshNumX}{#1}
\pgfmathtruncatemacro{\meshNumY}{#2}
\setlength{\meshBoxWidth}{#4}
\setlength{\meshBoxHeight}{#5}
\setlength{\meshLength}{#3}
\pgfmathsetmacro{\numXGapsMesh}{#1 - 1}
\pgfmathsetmacro{\numYGapsMesh}{#2 - 1}
\setlength{\meshXSize}{\numXGapsMesh \meshLength}
\setlength{\meshYSize}{\numYGapsMesh \meshLength}
\addtolength{\meshXSize}{\meshBoxWidth}
\addtolength{\meshYSize}{\meshBoxHeight}
\path (#6) coordinate (meshMidPoint);
\path (meshMidPoint) + (0,0.5\meshYSize) coordinate (meshTmp);
\path (meshTmp) + (-0.5\meshXSize,0) coordinate (meshTopLeft);	
\foreach \x in {1,...,#1}
  \foreach \y  in {1,...,#2}
 {	
  	\pgfmathtruncatemacro\xmm{\x - 1}
  	\pgfmathtruncatemacro\xpp{\x + 1}
 	\pgfmathtruncatemacro\ymm{\y - 1}
 	\pgfmathtruncatemacro\ypp{\y + 1}
	\path (meshTopLeft) + (\xmm \meshLength, -\ymm \meshLength) coordinate (meshPoint);
 	\defineBox{#8\x\y}{meshPoint}{#4}{#5}{#7}
 }
}

\newcommand{\textGrid}[5]{
%#1 number in x dim
%#2 number in y dim
%#3 grid size
%#4 middle of mesh
%#5 text
\pgfmathtruncatemacro{\meshNumX}{#1}
\pgfmathtruncatemacro{\meshNumY}{#2}
\setlength{\meshLength}{#3}
\pgfmathsetmacro{\numXGapsMesh}{#1 - 1}
\pgfmathsetmacro{\numYGapsMesh}{#2 - 1}
\setlength{\meshXSize}{\numXGapsMesh \meshLength}
\setlength{\meshYSize}{\numYGapsMesh \meshLength}
\path (#4) coordinate (meshMidPoint);
\path (meshMidPoint) + (0,0.5\meshYSize) coordinate (meshTmp);
\path (meshTmp) + (-0.5\meshXSize,0) coordinate (meshTopLeft);	
\foreach \x in {1,...,#1}
  \foreach \y  in {1,...,#2}
 {	
  	\pgfmathtruncatemacro\xmm{\x - 1}
  	\pgfmathtruncatemacro\xpp{\x + 1}
 	\pgfmathtruncatemacro\ymm{\y - 1}
 	\pgfmathtruncatemacro\ypp{\y + 1}
	\path (meshTopLeft) + (\xmm \meshLength, -\ymm \meshLength) coordinate (meshPoint);
	\node at (meshPoint) { {#5} };
 }
}