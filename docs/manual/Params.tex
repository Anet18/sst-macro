%% !TEX root = manual.tex

\chapter{Detailed Parameter Listings}
\label{chapter:parameters}
The following chapter is organized by parameter namespaces. Tables in each namespace are organized as
\def\arraystretch{1.5}%  1 is the default, change whatever you need

\begin{tabular}{\tableConfig}
\hline
Name (type) & Default if not given & Allowed \newline Values & Description \\
\hline
\end{tabular}

which lists the possible parameter names, allowed values, and brief descriptions.
More detailed descriptions of particular parameter values are found in the documentation in previous chapters.

The allowed parameter types are:

\begin{tabular}{| l | l |}
\hline
int & Any integer \\
\hline
long & Any integer value, but guaranteed not to overflow for long integers \\
\hline
bool & Either ``true'' or ``false'' as lowercase string \\
\hline
time & Any valid float value followed by time units (s,ms,us,ns,ps) \\
\hline
freq & Any valid float value followed by frequency units (Hz, MHz, GHz) \\
\hline
bandwidth & Any valid float value followed by bandwidth units (b/s, B/s, Mb/s, MB/s, etc) \\
\hline
byte length & Any positive integer followed by length units (B, KB, MB, GB,TB) \\
\hline
string & An arbitrary string \\
\hline
vector of X & A vector of type X with entries separated by spaces \\
\hline
filepath & A valid filepath to an \emph{existing} file, either absolute or relative \\
\hline
quantiy & A catch-all for a quantity with units. Any of frequency, bandwidth, byte length, or time can be given \\
\hline
\end{tabular}

\section{Global namespace}
\label{sec:globalParams}

\openTable
\hline
sst\_nthread \paramType{int} & 1 & Positive int & Only relevant for multi-threading. Specifying more threads than cores can lead to deadlock. \\
\hline
timestamp\_resolution \paramType{time} & 1ps & & Specifies the length of time occupied by 1 timestamp tick - the smallest resolvable time difference. Numerical stability depends on this parameter matching the time scales of the simulation. \\
\hline
serialization\_buffer\_size \paramType{byte length} & 512B & & Size to allocate for buffers in a pool of serialization buffers. This should set be large enough to handle serialization of most messages, but not so large that significant space is wasted. \\
\hline
cpu\_affinity \paramType{vector of int} & No default & Invalid cpu IDs give undefined behavior & When in multi-threading, specifies the list of core IDs that threads will be pinned to. \\
\hline
\end{tabular}

\section{Namespace ``topology''}
\label{sec:topologyParams}

\openTable
\hline
geometry \paramType{vector of int} & No default & See Topology section & Geometry configuration of the topology. For details of the keyword, users should refer to Section \ref{chapter:topologies} \\
\hline
name \paramType{string} & No default & torus, dragonfly, fat\_tree, butterfly, crossbar, tapered\_fat\_tree & The name of the topology to build. For details, see Section \ref{chapter:topologies} \\
\hline 
seed \paramType{long} & System time & & If no value given, random numbers for topology will be generated from system time \\
\hline
concentration \paramType{int} & 1 & Positive int & The number of nodes per network switch. For indirect networks, this is the number of nodes per leaf switch. \\
\hline
num\_leaf\_switches \paramType{int} & No default & Positive int & Only relevant for fat trees. This is the number of switches at the lowest level of the tree that are connected to compute nodes. Depending on how the fat tree is specified, this number may not be required. \\
\hline
\hline
k \paramType{int} & No default & int >= 2 & The branching fraction of a fat tree. k=2 is a binary tree. k=4 is a quad-tree. \\
\hline
l \paramType{int} & No default & Positive int & The number of levels in a fat tree. \\
\hline
num\_inj\_switches\_per\_subtree & No default & Positive int & For a tapered tree, the number of injection switches, $N_{inj}$, within an aggregation tree that connect directly toc ompute nodes. \\
\hline
num\_agg\_switches\_per\_subtree & No default & Positive int & For a tapered tree, the number of aggregations witches per aggregation tree linking injection switches to the core. \\
\hline
num\_agg\_subtrees & No default & Positive int & For a tapered fat tree with 3 levels (injection, aggregation, core), this gives the number, $N_{agg}$, of aggregation subtrees. To find the total number, $N_{tot}$ of injection (leaf) switches, we have $N_{tot} = N_{agg} \times N_{inj}$. \\
\hline
num\_core\_switches & No default & Positive int & The total number of core switches in a tapered tree linking the individual aggregation trees. \\
\hline
group\_connections \paramType{int} & No default & Positive int & For dragonfly, the number of intergroup connections on each switch in a Dragonfly group \\
\hline
redundant \paramType{vector of int} & vector of 1's & Positive ints & For Cartesian topologies (hypercube, dragonfly, torus) this specifies a bandwidth (redundancy) multiplier for network links in each dimension. \\
\hline
\end{tabular}

\section{Namespace ``node''}
\label{sec:nodeParams}

\openTable
\hline
model \paramType{string} & simple & simple & The type of node model (level of detail) for node-level operations \\
\hline
services \paramType{vector of strings} & Empty & Valid service names & For  details, see section on distributed services in developer's manual. Advanced feature. \\
\hline
\end{tabular}

\subsection{Namespace ``node.nic''}
\label{subsec:node:nic:Params}

\openTable
\hline
model \paramType{string} & No default & pisces, logP & The type of NIC model (level of detail) for modeling injection of messages (flows) to/from the network. \\
\hline
packetizer \paramType{string} & cut\_through & merlin, simple, cut\_through & The type of packetizer for injecting flows into the network. Merlin is part of sst-elements. Simple and cut-through use PISCES \\
\hline
packet\_allocator \paramType{string} & pisces & pisces & The packet object used by the NIC. Certain type of statistics or congestion models may require custom packets with extra fields. \\
\hline
negligible\_size \paramType{byte length} & 256B & & Messages (flows) smaller than size will not go through detailed congestion modeling. They will go through a simple analytic model to compute the delay. \\
\hline
\end{tabular}

\subsubsection{Namespace ``node.nic.delay\_histogram''}
\label{subsubsec:node:nic:delayHistogram:Params}
\openTable{8cm}
\hline
fileroot & string & No default & & 
  The path of a file to be created (or overwritten) where results from the statistics will be placed.  \\
\hline
bin\_size & quantiy & No default & Positive value & For the quantity being collected (bytes,time,etc), this determines how large a value range should be binned together. \\
\hline
num\_bins & int & 20 & Positive int & Determines the number of initial bins to allocate. If number is too small, simulation will allocate more. This is just a guess for optimization. \\
\hline
logarithmic & bool & false & & Whether to use a logarithmic axis for binning. \\
\hline
\end{tabular}


\subsubsection{Namespace ``node.nic.congestion\_spyplot''}
\label{subsubsec:node:nic:congestionSpyplot:Params}
\openTable{8cm}
\hline
fileroot & string & No default & & 
  The path of a file to be created (or overwritten) where results from the statistics will be placed.  \\
\hline
normalization & long & Ignored & Positive int & Determines a normalization value that will effect the color scales of an output PNG file. By default, if not specified, the spyplot will determine the largest value and normalize all values for the output PNG to be 0.0-1.0. Values larger than 1.0 will have the same color as 1.0. Useful for having two PNG files with different values have the same color scales. \\
\hline
\end{tabular}


\subsubsection{Namespace ``node.nic.traffic\_matrix''}
\label{subsubsec:node:nic:trafficMatrix:Params}
\openTable{8cm}
\hline
fileroot & string & No default & & 
  The path of a file to be created (or overwritten) where results from the statistics will be placed.  \\
\hline
normalization & long & Ignored & Positive int & Determines a normalization value that will effect the color scales of an output PNG file. By default, if not specified, the spyplot will determine the largest value and normalize all values for the output PNG to be 0.0-1.0. Values larger than 1.0 will have the same color as 1.0. Useful for having two PNG files with different values have the same color scales. \\
\hline
\end{tabular}


\subsubsection{Namespace ``node.nic.local\_bytes\_sent''}
\label{subsubsec:node:nic:localSent:Params}
\openTable
\hline
fileroot & string & No default & & 
  The path of a file to be created (or overwritten) where results from the statistics will be placed.  \\
\hline
\end{tabular}


\subsubsection{Namespace ``node.nic.global\_bytes\_sent''}
\label{subsubsec:node:nic:globalSent:Params}
\openTable
\hline
fileroot & string & No default & & 
  The path of a file to be created (or overwritten) where results from the statistics will be placed.  \\
\hline
\end{tabular}


\subsubsection{Namespace ``node.nic.message\_size\_histogram''}
\label{subsubsec:node:nic:sizeHistogram:Params}
\openTable{8cm}
\hline
fileroot & string & No default & & 
  The path of a file to be created (or overwritten) where results from the statistics will be placed.  \\
\hline
bin\_size & quantiy & No default & Positive value & For the quantity being collected (bytes,time,etc), this determines how large a value range should be binned together. \\
\hline
num\_bins & int & 20 & Positive int & Determines the number of initial bins to allocate. If number is too small, simulation will allocate more. This is just a guess for optimization. \\
\hline
logarithmic & bool & false & & Whether to use a logarithmic axis for binning. \\
\hline
\end{tabular}


\subsubsection{Namespace ``node.nic.ejection"}
\label{subsubsec:node:nic:ejection:Params}
These parameters do not need to be specified, but can be given.
Generally, the simulation assumes an infinite buffer size (unlimited memory) and no latency.
All other parameters can be filled in from \inlinefile{node.nic.injection}.

\openTable
\hline
stats \paramType{string} & null & 
	delay\_histogram,
  congestion\_spyplot,
  multi,
	null 
 & The type of statistics to collect from packets leaving or arriving at the NIC.
   Null indicates no statistics collected. For details on the other types of statistics, see Section \ref{sec:tutorials:packetStats} \\
\hline
callbacks \paramType{vector of string} & No default & Any valid stats name & If \inlinefile{stats} is set to \inlinefile{multi},then multiple different stats can be given here \\
\hline
arbitrator \paramType{string} & cut\_through & null, simple, cut\_through & Bandwidth arbitrator for PISCES congestion modeling. Null uses simple delays with no congestion. Simple uses store-and-forward that is cheap to compute, but can have severe latency errors for large packets. Cut-through approximates pipelining of flits across stages.  \\
\hline
latency \paramType{time} & No default & & If given, overwrites the send and credit latency parameters. Depending on component, the entire latency may be put on either the credits or the send. \\
\hline
bandwidth & No default & & The bandwidth of the arbitrator \\
\hline
send\_latency \paramType{time} & No default & & The latency to send a packet to the next stage in the network. This can be omitted if the generic latency parameter is given (see above). \\
\hline
credit\_latency \paramType{time} & No default & & The latency to send a credit to the previous network stage. This can be omitted if the generic latency parameter is given (see above). \\
\hline
credits \paramType{byte length} & No default & & The number of initial credits for the component. Corresponds to an input buffer on another component. In many cases, SST/macro can compute this from other parameters and fill in the value. In some cases, it will be required. \\
\hline
mtu \paramType{byte length} & 1024B & & The packet size. All messages (flows) will be broken into units of this size. \\

 
\hline
\end{tabular}

\subsubsection{Namespace ``node.nic.injection"}
\label{subsubsec:node:nic:injection:Params}

\openTable
\hline
stats \paramType{string} & null & 
	delay\_histogram,
	null 
 & The type of statistics to collect from packets leaving or arriving at the NIC.
   Null indicates no statistics collected. For details on the other types of statistics, see Section \ref{sec:tutorials:packetStats} \\
arbitrator \paramType{string} & cut\_through & null, simple, cut\_through & Bandwidth arbitrator for PISCES congestion modeling. Null uses simple delays with no congestion. Simple uses store-and-forward that is cheap to compute, but can have severe latency errors for large packets. Cut-through approximates pipelining of flits across stages.  \\
\hline
latency \paramType{time} & No default & & If given, overwrites the send and credit latency parameters. Depending on component, the entire latency may be put on either the credits or the send. \\
\hline
bandwidth & No default & & The bandwidth of the arbitrator \\
\hline
send\_latency \paramType{time} & No default & & The latency to send a packet to the next stage in the network. This can be omitted if the generic latency parameter is given (see above). \\
\hline
credit\_latency \paramType{time} & No default & & The latency to send a credit to the previous network stage. This can be omitted if the generic latency parameter is given (see above). \\
\hline
credits \paramType{byte length} & No default & & The number of initial credits for the component. Corresponds to an input buffer on another component. In many cases, SST/macro can compute this from other parameters and fill in the value. In some cases, it will be required. \\
\hline
mtu \paramType{byte length} & 1024B & & The packet size. All messages (flows) will be broken into units of this size. \\

 
\hline
\end{tabular}


\subsection{Namespace ``node.memory''}
\label{subsec:node:memory:Params}

\openTable
\hline
model \paramType{string} & No default & logP, pisces & The type of memory model (level of detail) for modeling memory transactions. \\
\hline
arbitrator \paramType{string} & cut\_through & null, simple, cut\_through & The type of arbitrator. See arbitrator descriptions above. \\
\hline
latency \paramType{time} & No default & & The latency of single memory operation \\
\hline
total\_bandwidth & No default &  & The total memory bandwidth possible across all memory controllers. \\
\hline
max\_single\_bandwidth & Computed & & The maximum memory bandwidth of a single stream of requests. Essentially the bandwidth of a single memory controller. If not given, this defaults the value of total\_bandwidth. \\
\hline
\end{tabular}

\subsection{Namespace ``node.os"}
\label{subsec:node:os:Params}

\openTable
\hline
compute\_scheduler \paramType{string} & simple & simple, cpuset & The level of detail for scheduling compute tasks to cores. Simple looks for any empty core. cpuset allows bitmasks to be set for defining core affinities. \\
\hline
stack\_size \paramType{byte length} & 32KB & & The size of user-space thread stack to allocate for each virtual application \\
\hline
stack\_protect \paramType{bool} & false & & Whether to put special mprotect regions around the thread stacks to turn stack overflow errors into seg faults \\
\hline
stack\_chunk\_size \paramType{byte length} & 1MB & & The size of memory to allocate at a time when allocating new thread stacks. Rather than allocating one thread stack at a time, multiple stacks are allocated and added to a pool as needed.  \\
\hline
\end{tabular}

\subsubsection{Namespace ``node.os.call\_graph"}
\label{subsubsec:node:os:callGraph:Params}
\openTable
\hline
fileroot & string & No default & & 
  The path of a file to be created (or overwritten) where results from the statistics will be placed.  \\
\hline
\end{tabular}

\subsubsection{Namespace ``node.os.ftq"}
\label{subsubsec:node:os:ftq:Params}
\openTable
\hline
fileroot & string & No default & & 
  The path of a file to be created (or overwritten) where results from the statistics will be placed.  \\
\hline 
epoch \paramType{time} & No default & & The length of simulation time to treat as a single interval. This is much like a histogram bin, except the bin is partitioned into multiple categories. If too small, not enough data will be collected per interval to give reasonable looking results. If too large, significant changes in system activity over time will not be resolved. \\
\hline
\end{tabular}

\subsection{Namespace ``node.proc''}
\label{subsec:node:proc:Params}
\openTable
\hline
ncores \paramType{int} & No default & Positive int & The number of cores contained in a processor (socket). Total number of cores for a node is $ncores \times nsockets$. \\
\hline
frequency & No default & & The baseline frequency of the node \\
\hline
parallelism \paramType{double} & 1.0 & Positive number & Fudge factor to account for superscalar processor. Number of flops per cycle performed by processor. \\
\hline
\end{tabular}

\section{Namespace ``netlink''}
\label{sec:netlink:Params}

\openTable
\hline
model \paramType{string} & null & null, pisces & The type of netlink congestion model to use. If null, no netlink blocks will be used.  \\
\hline
concentration \paramType{int} & No default & Positive int & The number of nodes connected to a given netlink. The node concentration per switch must be greater than or equal to the node concentration per netlink. \\
\hline
num\_tiles \paramType{int} & 1 & Positive int & The number of router tiles the netlink connects to. Only relevant for tiled router models. Advanced usage. \\
\hline
\end{tabular}

\subsection{Namespace ``netlink.injection"}
\label{subsec:netlink:injection:Params}
\openTable{8cm}
\hline
arbitrator \paramType{string} & cut\_through & null, simple, cut\_through & Bandwidth arbitrator for PISCES congestion modeling. Null uses simple delays with no congestion. Simple uses store-and-forward that is cheap to compute, but can have severe latency errors for large packets. Cut-through approximates pipelining of flits across stages.  \\
\hline
latency \paramType{time} & No default & & If given, overwrites the send and credit latency parameters. Depending on component, the entire latency may be put on either the credits or the send. \\
\hline
bandwidth & No default & & The bandwidth of the arbitrator \\
\hline
send\_latency \paramType{time} & No default & & The latency to send a packet to the next stage in the network. This can be omitted if the generic latency parameter is given (see above). \\
\hline
credit\_latency \paramType{time} & No default & & The latency to send a credit to the previous network stage. This can be omitted if the generic latency parameter is given (see above). \\
\hline
credits \paramType{byte length} & No default & & The number of initial credits for the component. Corresponds to an input buffer on another component. In many cases, SST/macro can compute this from other parameters and fill in the value. In some cases, it will be required. \\
\hline
mtu \paramType{byte length} & 1024B & & The packet size. All messages (flows) will be broken into units of this size. \\


\hline
\end{tabular}


\subsection{Namespace ``netlink.ejection"}
\label{subsec:netlink:ejection:Params}
\openTable{8cm}
\hline
arbitrator \paramType{string} & cut\_through & null, simple, cut\_through & Bandwidth arbitrator for PISCES congestion modeling. Null uses simple delays with no congestion. Simple uses store-and-forward that is cheap to compute, but can have severe latency errors for large packets. Cut-through approximates pipelining of flits across stages.  \\
\hline
latency \paramType{time} & No default & & If given, overwrites the send and credit latency parameters. Depending on component, the entire latency may be put on either the credits or the send. \\
\hline
bandwidth & No default & & The bandwidth of the arbitrator \\
\hline
send\_latency \paramType{time} & No default & & The latency to send a packet to the next stage in the network. This can be omitted if the generic latency parameter is given (see above). \\
\hline
credit\_latency \paramType{time} & No default & & The latency to send a credit to the previous network stage. This can be omitted if the generic latency parameter is given (see above). \\
\hline
credits \paramType{byte length} & No default & & The number of initial credits for the component. Corresponds to an input buffer on another component. In many cases, SST/macro can compute this from other parameters and fill in the value. In some cases, it will be required. \\
\hline
mtu \paramType{byte length} & 1024B & & The packet size. All messages (flows) will be broken into units of this size. \\


\hline
\end{tabular}


\section{Namespace ``mpi"}
\label{sec:mpi:Params}

\openTable
\hline
test\_delay \paramType{time} & 0 & & The minimum time spent by MPI on each MPI\_Test call \\
\hline
iprobe\_delay \paramType{time} & 0 & & The minimum time spent by MPI on each MPI\_Iprobe call \\
\hline
\end{tabular}

\subsection{Namespace ``mpi.queue''}
\label{subsec:mpi:queue:Params}

\openTable
\hline
max\_vshort\_msg\_size \paramType{byte length} & 512B & & The maximum size to use the very short message protocol. Small messages are sent eagerly using special pre-allocated mailbox buffers. Sends complete immediately. \\
\hline
max\_eager\_msg\_size \paramType{byte length} & 8KB & & The maximum size to use the RDMA eager protocol. This also uses buffers to send message, but instead of using pre-allocated mailboxes, it coordinates an RDMA get. Sends complete immediately. \\
\hline
post\_rdma\_delay \paramType{time} & 0 & & The minimum time spent by MPI posting each RDMA operation \\
\hline
post\_header\_delay \paramType{time} & 0 & & The mimimum time spent by MPI sending headers into pre-allocated mailboxes \\
\hline
poll\_delay (time) & 0 & & The minimum time spent by MPI each time it polls for incoming messages \\
\hline
\end{tabular}

\section{Namespace ``switch''}
\label{subsec:switch:Params}

\openTable
\hline
model \paramType{string} & No default & logP, pisces & The type of switch model (level of detail) for modeling network traffic. \\
\hline
buffer\_size \paramType{byte length} & No default & & The size of input and output buffers on each switch. This determines the number of credits available to other components \\
\hline
\end{tabular}

\subsection{Namespace ``switch.router''}
\label{subsec:switch:router:Params}

\openTable
\hline
name \paramType{string} & No default & minimal, valiant, ugal, fat\_tree & The name of the routing algorithm to use for routing packets. \\
\hline
ugal\_threshold \paramType{int} & 0 & & The minimum number of network hops required before UGAL is considered. All path lengths less than value automatically use minimal. \\
\hline
\end{tabular}

\subsection{Namespace ``switch.output\_buffer"}
\label{subsec:switch:outputBuffer:Params}

\openTable
\hline
stats \paramType{string} & null & 
	delay\_histogram,
  byte\_hops,
	null 
 & The type of statistics to collect from packets leaving or arriving at the switch output buffers.
   Null indicates no statistics collected. For details on the other types of statistics, see Section \ref{sec:tutorials:packetStats} \\
\hline
\end{tabular}

\subsubsection{Namespace ``switch.output\_buffer.delay\_histogram''}
\label{subsubsec:switch:outputBuffer:delayHistogram:Params}
\openTable{8cm}
\hline
fileroot & string & No default & & 
  The path of a file to be created (or overwritten) where results from the statistics will be placed.  \\
\hline
bin\_size & quantiy & No default & Positive value & For the quantity being collected (bytes,time,etc), this determines how large a value range should be binned together. \\
\hline
num\_bins & int & 20 & Positive int & Determines the number of initial bins to allocate. If number is too small, simulation will allocate more. This is just a guess for optimization. \\
\hline
logarithmic & bool & false & & Whether to use a logarithmic axis for binning. \\
\hline
\end{tabular}


\subsubsection{Namespace ``switch.output\_buffer.byte\_hops''}
\label{subsubsec:switch:outputBuffer:delayHistogram:Params}
\openTable
\hline
fileroot & string & No default & & 
  The path of a file to be created (or overwritten) where results from the statistics will be placed.  \\
\hline
\end{tabular}


\subsection{Namespace ``switch.xbar"}
\label{sec:switch:outputBuffer:delayHistogram:Params}

\openTable
\hline
stats \paramType{string} & null & 
	bytes\_sent,
	null 
 & The type of statistics to collect from packets leaving or arriving at the switch crossbar.
   Null indicates no statistics collected. For details on the other types of statistics, see Section \ref{sec:tutorials:packetStats} \\
\hline
arbitrator \paramType{string} & cut\_through & null, simple, cut\_through & Bandwidth arbitrator for PISCES congestion modeling. Null uses simple delays with no congestion. Simple uses store-and-forward that is cheap to compute, but can have severe latency errors for large packets. Cut-through approximates pipelining of flits across stages.  \\
\hline
latency \paramType{time} & No default & & If given, overwrites the send and credit latency parameters. Depending on component, the entire latency may be put on either the credits or the send. \\
\hline
bandwidth & No default & & The bandwidth of the arbitrator \\
\hline
send\_latency \paramType{time} & No default & & The latency to send a packet to the next stage in the network. This can be omitted if the generic latency parameter is given (see above). \\
\hline
credit\_latency \paramType{time} & No default & & The latency to send a credit to the previous network stage. This can be omitted if the generic latency parameter is given (see above). \\
\hline
credits \paramType{byte length} & No default & & The number of initial credits for the component. Corresponds to an input buffer on another component. In many cases, SST/macro can compute this from other parameters and fill in the value. In some cases, it will be required. \\
\hline
mtu \paramType{byte length} & 1024B & & The packet size. All messages (flows) will be broken into units of this size. \\


\hline
\end{tabular}

\subsubsection{Namespace ``switch.xbar.bytes\_sent''}
\label{subsubsec:switch:xbar:bytesSent:Params}
\openTable
\hline
fileroot & string & No default & & 
  The path of a file to be created (or overwritten) where results from the statistics will be placed.  \\
\hline
\end{tabular}


\subsection{Namespace ``switch.link''}
\label{subsec:switch:link:Params}
\openTable{8cm}
\hline
arbitrator \paramType{string} & cut\_through & null, simple, cut\_through & Bandwidth arbitrator for PISCES congestion modeling. Null uses simple delays with no congestion. Simple uses store-and-forward that is cheap to compute, but can have severe latency errors for large packets. Cut-through approximates pipelining of flits across stages.  \\
\hline
latency \paramType{time} & No default & & If given, overwrites the send and credit latency parameters. Depending on component, the entire latency may be put on either the credits or the send. \\
\hline
bandwidth & No default & & The bandwidth of the arbitrator \\
\hline
send\_latency \paramType{time} & No default & & The latency to send a packet to the next stage in the network. This can be omitted if the generic latency parameter is given (see above). \\
\hline
credit\_latency \paramType{time} & No default & & The latency to send a credit to the previous network stage. This can be omitted if the generic latency parameter is given (see above). \\
\hline
credits \paramType{byte length} & No default & & The number of initial credits for the component. Corresponds to an input buffer on another component. In many cases, SST/macro can compute this from other parameters and fill in the value. In some cases, it will be required. \\
\hline
mtu \paramType{byte length} & 1024B & & The packet size. All messages (flows) will be broken into units of this size. \\


\hline
\end{tabular}


\subsection{Namespace ``switch.ejection''}
\label{subsec:switch:ejection:Params}
This namespace is not actually required. 
If unspecified, all of the values here will be filled in from \inlinefile{nic.injection}.

\openTable{8cm}
\hline
arbitrator \paramType{string} & cut\_through & null, simple, cut\_through & Bandwidth arbitrator for PISCES congestion modeling. Null uses simple delays with no congestion. Simple uses store-and-forward that is cheap to compute, but can have severe latency errors for large packets. Cut-through approximates pipelining of flits across stages.  \\
\hline
latency \paramType{time} & No default & & If given, overwrites the send and credit latency parameters. Depending on component, the entire latency may be put on either the credits or the send. \\
\hline
bandwidth & No default & & The bandwidth of the arbitrator \\
\hline
send\_latency \paramType{time} & No default & & The latency to send a packet to the next stage in the network. This can be omitted if the generic latency parameter is given (see above). \\
\hline
credit\_latency \paramType{time} & No default & & The latency to send a credit to the previous network stage. This can be omitted if the generic latency parameter is given (see above). \\
\hline
credits \paramType{byte length} & No default & & The number of initial credits for the component. Corresponds to an input buffer on another component. In many cases, SST/macro can compute this from other parameters and fill in the value. In some cases, it will be required. \\
\hline
mtu \paramType{byte length} & 1024B & & The packet size. All messages (flows) will be broken into units of this size. \\


\hline
\end{tabular}


\section{Namespace ``appN''}
\label{sec:appN:Params}
This is a series of namespaces \inlineshell{app1}, \inlineshell{app2}, and so on for each of the launched applications. These should be contained within the \inlineshell{node} namespace.

\openTable
\hline
name \paramType{string} & No default & parsedumpi, cxx\_full\_main, cxx\_empty\_main & The name of the application to launch. Very few applications are built-in. Registration of external apps is shown starting in Section \ref{sec:tutorial:basicmpi}. \\
\hline
size \paramType{int} & No default & Positive int & The number of procs (MPI ranks) to launch. If launch\_cmd given, this parameter is not required. \\
\hline
start \paramType{int} & 0 & & The time at which a launch request for the application will be made \\
\hline
concentration \paramType{int} & 1 & Positive int & The number of procs (MPI ranks) per compute node \\
\hline
core\_affinities \paramType{vector of int} & Empty & & \\
\hline
launch\_cmd \paramType{string} & No default & Valid aprun or srun & This uses a launch command as would be found with ALPS or SLURM launchers on real systems, e.g. aprun -n 4 -N 1 \\
\hline
indexing \paramType{string} & block & block, random, cart, node\_id, coordinate & The indexing scheme for assign proc ID (MPI rank number) to compute nodes \\
\hline
node\_id\_mapper\_file \paramType{filepath} & No default & & If using Node ID indexing, the file containing the node ID index list \\
\hline 
random\_indexer\_seed \paramType{long} & System time & & The seed to use for a random allocation. If not specified, system time is used. \\
\hline
allocation \paramType{string} & first\_available & first\_available, random, cart, node\_id, coordinate & The scheme to use for allocating compute nodes to a given job. \\
\hline
random\_allocation\_seed \paramType{long} & System time & & For random allocation policy. If unspecified, system time is used as the seed.  \\
\hline
node\_id\_allocation\_file \paramType{filepath} & No default & & If using Node ID allocation, the file containing the list of node IDs to allocate for the job \\
\hline
dumpi\_metaname \paramType{filepath} & No default & & If running DUMPI trace, the location of the metafile for configuring trace replay \\ 
\hline
coordinate\_file \paramType{filepath} & No default & & If running using coordinate allocation or indexing, the path to the file containing the node coordinates of each proc (MPI rank) \\ 
\hline
cart\_sizes \paramType{vector of int} & No default & & Launch a contiguous block of nodes in a Cartesian topology. This gives the size of each dimension in the block. \\
\hline 
cart\_offsets \paramType{vector of int} & No default & & Launch a contiguous block nodes in a Cartesian topology. This gives the offset in each dimension where the block begins. \\
\hline
parsedumpi\_timescale \paramType{double} & 1.0 & Positive float & If running DUMPI traces, scale compute times by the given value. Values less than 1.0 speed up computation. Values greater than 1.0 slow down computation. \\
\hline
parsedumpi\_terminate\_percent \paramType{int} & 100 & 1-100 & Percent of trace. Can be used to terminate large traces early \\
\hline
host\_compute\_timer \paramType{bool} & False & & Use the compute time on the host to estimate compute delays \\
\hline
\end{tabular}

\openTable
\\
\hline
otf2\_metafile \paramType{string} & No default & string & The root file of an OTF2 trace. \\
\hline
otf2\_timescale \paramType{double} & 1.0 & Positive float & If running OTF2 traces, scale compute times by the given value. Values less than 1.0 speed up computation. Values greater than 1.0 slow down computation. \\
\hline
otf2\_print\_mpi\_calls \paramType{bool} & false & & Print MPI calls found in the OTF2 trace
\\
\hline
otf2\_print\_trace\_events \paramType{bool} & false & & Debugging flag that printsindividual trace events (which includes details such as when an MPI call begins, ends, and when a collective begins and ends
\\
\hline
otf2\_print\_time\_deltas \paramType{bool} & false & & Debugging flag that prints compute delays injected by the simulator
\\
\hline
otf2\_warn\_unknown\_callback \paramType{bool} & false & & Debugging flag the prints unknown callbacks
\\
\hline
\end{tabular}


