% !TEX root = developer.tex

\chapter{Introduction}\label{chapter:intro}

\section{Use of C++/Boost}\label{sec:boost}
SST/macro (Structural Simulation Toolkit for Macroscale) is a discrete event simulator designed for macroscale (system-level) experiments in HPC. 
SST/macro is an object-oriented C++ code that makes heavy use of dynamic types and polymorphism.
While a great deal of template machinery exists under the hood, nearly all users and even most developers will never actually need to interact with any C++ templates.
Most template wizardry is hidden in easy-to-use macros.
While C++ allows a great deal of flexibility in syntax and code structure, we are striving towards a unified coding style.

\section{Polymorphism and Modularity}\label{sec:polymorphism}
The simulation progresses with different modules (classes) exchanging messages.
In general, when module 1 sendings a message to module 2, module 1 only sees an abstract interface for module 2.
The polymorphic type of module 2 can vary freely to employ different physics or congestions models without affecting the implementation of module 1. 
Polymorphism, while greatly simplifying modularity and interchangeability, does have some consequences.
The ``workhorse'' of SST/macro is the base \inlinecode{event} and \inlinecode{message} classes.
To increase polymorphism and flexibility, every SST/macro module that receives messages does so via the generic function

\begin{CppCode}
void
handle(event* ev){
...
}
\end{CppCode}
The prototype therefore accepts \emph{any} event type. The class \inlinecode{message} is a special type of event that refers specifically to a message (e.g. MPI message) carrying a data payload.
Misusing types in SST/macro is \emph{not} a compile-time error.
The onus of correct event types falls on runtime assertions.
All event types may not be valid for a given module.
A module for the memory subsystem should throw an error if the developer accidentally passes it a event intended for the OS or the NIC.
Efforts are being made to convert runtime errors into compile-time errors.
In many cases, though, this cannot be avoided.
The other consequence is that a lot of dynamic casts appear in the code.
An abstract \inlinecode{event} type is received, but must be converted to the specific message type desired.
NOTE: While \emph{some} dynamic casts are sometimes very expensive in C++ (and are implementation-dependent),
most SST/macro dynamic casts are simple equality tests involving virtual table pointers.

While, SST/macro strives to be as modular as possible, allowing arbitrary memory, NIC, interconnect components,
in many cases certain physical models are simply not compatible.
For example, using a fluid flow model for memory reads cannot be easily combined with a packet-based model for the network.
Again, pairing incompatible modules is not a compile-time error.
Only when the types are fully defined at runtime can an incompatibility error be detected.
Again, efforts are being made to convert as many type-usage problems into compiler errors.
We prefer simulation flexibility to compiler strictness, though. 

\section{Most Important Style and Coding Rules}\label{sec:stylerules}
Here is a selection of C++ rules we have tended to follow.  
Detailed more below, example scripts are included to reformat the code style however you may prefer for local editing.
However, if committing changes to the repository, only use the default formatting in the example scripts.
\begin{itemize}
\item snake\_case is used for variable and class names.
\item We use ``one true brace`` style (OTBS) for source files. 
\item In header files, all functions are inline style with attached brackets.
\item To keep code compact horizontally, indent is set to two spaces. Namespaces are not indented.
\item Generally, all if-else and for-loops have brackets even if a single line.
\item Accessor functions are not prefixed, i.e. a function would be called \inlinecode{name()} not \inlinecode{get_name()}, except
where conflicts require a prefix. Functions for modifying variables are prefixed with \inlinecode{set_},
\item We use .h and .cc instead of .hpp and .cpp
\item As much implementation as possible should go in .cc files.
	Header files can end up in long dependency lists in the make system.  
	Small changes to header files can result in long recompiles.
	If the function is more than a basic set/get, put it into a .cc file.
\item Header files with many classes are discouraged.  When reasonable, one class per header file/source file is preferred.
	 Many short files are better than a few really long ones.
\item Document, document, document.  If it isn't obvious what a function does, add doxygen-compatible documentation.
	Examples are better than abstract wording.
\item Use STL and Boost containers for data structures.  Do not worry about performance until much later.
	Premature optimization is the root of all evil. If determined that an optimized data structure is needed,
	please do that after the entire class is complete and debugged.
\item Forward declarations.  There are a lot of interrelated classes, often creating circular dependencies. In addition, you can add up with an explosion of include files, often involving classes that are never used in a given \inlineshell{.cc} file.  For cleanliness of compilation, you will find many \inlineshell{*_fwd.h} files throughout the code. If you need a forward declaration, we encourage including this header file rather than ad hoc forward declarations throughout the code.
\end{itemize}

\section{Code Style Reformatting}\label{sec:styleReformat}
Since we respect the sensitivity of code-style wars, we include scripts that demonstrate basic usage of the C++ code formatting tool \emph{astyle}.
This can be downloaded from {http://astyle.sourceforge.net}. 
A python script called \inlineshell{fix_style} is included in the top-level bin directory.
It recursively reformats all files in a given directory and its subdirectories.

\section{Memory Allocation}\label{sec:memalloc}
To improve performance, custom memory allocation strategies can be employed in C++.
At present, a global custom \inlinecode{operator new} can be optionally activated which is optimized for large pages and memory pools.
At present, no class-specific implementation of \inlinecode{operator new} is used.
However, we may soon implement custom allocation techniques to improve things like cache/TLB efficiency.
This is the only major change expected to SST/macro that would affect externally developed modules - and even here the expected modifications would be quite small.
Because custom allocation schemes may be used, all externally developed code should use \inlinecode{operator new}, rather than \emph{not}  \inlinecode{malloc} or \inlinecode{mmap}, unless allocating very large arrays.

