% !TEX root = developer.tex

\chapter{A Custom Object: Beginning To End}
\label{chapter:custom}

Suppose we have brilliant design for a new topology we want to test.
We want to run a simple test \emph{without} having to modify the \sstmacro build system.
We can create a simple external project that links the new topology object to \sstmacro libraries.
The Makefile can be found in \inlineshell{tutorials/programming/topology}.
You are free to make \emph{any} Makefile you want.
After \sstmacro installs, it creates compiler wrappers \inlineshell{sst++} and \inlineshell{sstcc}
in the chosen \inlineshell{bin} folder.  
These are essentially analogs of the MPI compiler wrappers.
This configures all include and linkage for the simulation.

We want to make an experimental topology in a ring.
Rather than a simple ring with connections to nearest neighbors, though, we will have ``express'' connections that jump to switches far away.

We begin with the standard typedefs.

\begin{CppCode}
#include <sstmac/hardware/topology/structured_topology.h>

namespace sstmac {
namespace hw {

class xpress_ring :
  public structured_topology
{
 public:
  typedef enum {
    up_port = 0,
    down_port = 1,
    jump_up_port = 2,
    jump_down_port = 3
  } port_t;

  typedef enum {
    jump = 0, step = 1
  } stride_t;

\end{CppCode} 
Packets can either go to a nearest neighbor or they can ``jump'' to a switch further away.
Each switch in the topology will need four ports for step/jump going up/down.
The header file can be found in \inlineshell{tutorials/programm/topology/xpressring.h}.
We now walk through each of the functions in turn in the source in the topology public interface.
We got some functions for free by inheriting from \inlinecode{structured_topology}.

We start with

\begin{CppCode}
xpress_ring::xpress_ring(sprockit::sim_parameters* params) :
  structured_topology(params)
{
  ring_size_ = params->get_int_param("xpress_ring_size");
  jump_size_ = params->get_int_param("xpress_jump_size");
}
\end{CppCode}
determining how many switches are in the ring and how big a ``jump'' link is.

The topology then needs to tell objects how to connect

\begin{CppCode}
void xpress_ring::connectedOutports(SwitchId src, std::vector<connection>& conns)
{
  conns.resize(4); //every switch has 4 connections
  auto& plusConn = conns[0];
  plusConn.src = src;
  plusConn.dst = (src+1) % ring_size_;
  plusConn.src_outport = 0;
  plusConn.dst_inport = 1; 
  
  auto& minusConn = conns[1];
  minusConn.src = src;
  minusConn.dst = (src -1 + ring_size) % ring_size_;
  minusConn.src_outport = 1;
  minusConn.dst_inport = 0;   
  
  auto& jumpUpConn = conns[2];
  jumpUpConn.src = src;
  jumpUpConn.dst = (src + jump_size_) % ring_size_;
  jumpUpConn.src_outport = 2;
  jumpUpConn.dst_inport = 3;
  
  auto& jumpDownConn = conns[2];
  jumpDownConn.src = src;
  jumpDownConn.dst = (src - jump_size_ + ring_size) % ring_size_;
  jumpDownConn.src_outport = 3;
  jumpDownConn.dst_inport = 2;
}
\end{CppCode}
Each of the four connections get a different unique port number.  
We must identify both the outport port for the sender and the input port for the receiver.

To compute the distance between two switches

\begin{CppCode}
int xpress_ring::num_hops(int total_distance) const
{
  int num_jumps = total_distance / jump_size_;
  int num_steps = total_distance % jump_size_;
  int half_jump = jump_size_ / 2;
  if (num_steps > half_jump) {
    //take an extra jump
    ++num_jumps;
    num_steps = jump_size_ - num_steps;
  }
  return num_jumps + num_steps;
}

int
xpress_ring::minimalDistance(
  const coordinates& src_coords,
  const coordinates& dest_coords) const
{
  int src_pos = src_coords[0];
  int dest_pos = dest_coords[0];
  int up_distance = abs(dest_pos - src_pos);
  int down_distance = abs(src_pos + ring_size_ - dest_pos);

  int total_distance = std::max(up_distance, down_distance);
  return num_hops(total_distance);
}
\end{CppCode}
Essentially you compute the number of jumps to get close to the final destination and then the number of remaining single steps.

We are now ready to use our topology in an application.
In this case, we just demo with the built-in MPI ping all program from \sstmacro.
Here every node in the network sends a point-to-point message to every other node.
There is a parameter file in the \inlineshell{tutorials/programming/toplogy} folder.
To specify the new topology

\begin{ViFile}
# Topology
topology.name = xpress
topology.xpress_ring_size = 10
topology.xpress_jump_size = 5
\end{ViFile}



