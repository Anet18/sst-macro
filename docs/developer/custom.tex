% !TEX root = developer.tex

\chapter{A Custom Object: Beginning To End}
\label{chapter:custom}

Suppose we have brilliant design for a new topology we want to test.
We want to run a simple test \emph{without} having to modify the \sstmacro build system.
We can create a simple external project that links the new topology object to \sstmacro libraries.
The Makefile can be found in \inlineshell{tutorials/programming/topology}.
You are free to make \emph{any} Makefile you want.
After \sstmacro installs, it creates compiler wrappers \inlineshell{sst++} and \inlineshell{sstcc}
in the chosen \inlineshell{bin} folder.  
These are essentially analogs of the MPI compiler wrappers.
This configures all include and linkage for the simulation.

We want to make an experimental topology in a ring.
Rather than a simple ring with connections to nearest neighbors, though, we will have ``express'' connections that jump to switches far away.

We begin with the standard typedefs.
\begin{CppCode}
#include <sstmac/hardware/topology/structured_topology.h>

namespace sstmac {
namespace hw {

class xpress_ring :
  public structured_topology
{
 public:
  typedef enum {
    up_port = 0,
    down_port = 1,
    jump_up_port = 2,
    jump_down_port = 3
  } port_t;

  typedef enum {
    jump = 0, step = 1
  } stride_t;

\end{CppCode} 
Packets can either go to a nearest neighbor or they can ``jump'' to a switch further away.
Each switch in the topology will need four ports for step/jump going up/down.
The header file can be found in \inlineshell{tutorials/programm/topology/xpressring.h}.
We now walk through each of the functions in turn in the source in the topology public interface.
We got some functions for free by inheriting from \inlinecode{structured_topology}.

We start with

\begin{CppCode}
void
xpress_ring::init_factory_params(sprockit::sim_parameters* params)
{
  structured_topology::init_factory_params(params);
  ring_size_ = params->get_int_param("xpress_ring_size");
  jump_size_ = params->get_int_param("xpress_jump_size");
}
\end{CppCode}
determining how many switches are in the ring and how big a ``jump'' link is.

The topology then needs to tell objects how to connect

\begin{CppCode}
void
xpress_ring::connect_objects(connectable_map& objects)
{
  for (int i=0; i < ring_size_; ++i) {
    connectable* center_obj = objects[switch_id(i)];

    switch_id up_idx((i + 1) % ring_size_);
    connectable* up_partner = find_object(objects, cloner, up_idx);
    center_obj->connect(up_port, down_port, connectable::network_link, up_partner);

    switch_id down_idx((i + ring_size_ - 1) % ring_size_);
    connectable* down_partner = find_object(objects, cloner, down_idx);
    center_obj->connect_mod_at_port(down_port, up_port, connectable::network_link,
                                    down_partner);

    switch_id jump_up_idx((i + jump_size_) % ring_size_);
    connectable* jump_up_partner = find_object(objects, cloner, jump_up_idx);
    center_obj->connect(jump_up_port, jump_down_port, connectable::network_link,
                                    jump_up_partner);

    switch_id jump_down_idx((i + ring_size_ - jump_size_) % ring_size_);
    connectable* jump_down_partner = find_object(objects, cloner,
                                         jump_down_idx);
    center_obj->connect(jump_down_port, jump_up_port, connectable::network_link,
                                    jump_down_partner);
  }
}
\end{CppCode}
We loop through every switch in the ring and form $+/-$ connections to neighbors and $+/-$ connections to jump partners.
Each of the four connections get a different unique port number.  We must identify both the outport port for the sender and the input port for the receiver.

To compute the distance between two switches

\begin{CppCode}
int
xpress_ring::num_hops(int total_distance) const
{
  int num_jumps = total_distance / jump_size_;
  int num_steps = total_distance % jump_size_;
  int half_jump = jump_size_ / 2;
  if (num_steps > half_jump) {
    //take an extra jump
    ++num_jumps;
    num_steps = jump_size_ - num_steps;
  }
  return num_jumps + num_steps;
}

int
xpress_ring::minimal_distance(
  const coordinates& src_coords,
  const coordinates& dest_coords) const
{
  int src_pos = src_coords[0];
  int dest_pos = dest_coords[0];
  int up_distance = abs(dest_pos - src_pos);
  int down_distance = abs(src_pos + ring_size_ - dest_pos);

  int total_distance = std::max(up_distance, down_distance);
  return num_hops(total_distance);
}
\end{CppCode}
Essentially you compute the number of jumps to get close to the final destination and then the number of remaining single steps.

For computing coordinates, the topology has dimension one.
\begin{CppCode}
switch_id
xpress_ring::get_switch_id(const coordinates& coords) const
{
  return switch_id(coords[0]);
}

void
xpress_ring::get_productive_path(
  int dim,
  const coordinates& src,
  const coordinates& dst,
  routable::path& path) const
{
  minimal_route_to_coords(src, dst, path);
}

void
xpress_ring::compute_switch_coords(switch_id swid, coordinates& coords) const
{
  coords[0] = int(swid);
}
\end{CppCode}
Thus the coordinate vector is a single element with the \switchid.

The most complicated function is the routing function.

\begin{CppCode}
void
xpress_ring::minimal_route_to_coords(
  const coordinates& src_coords,
  const coordinates& dest_coords,
  routable::path& path) const
{
  int src_pos = src_coords[0];
  int dest_pos = dest_coords[0];

  //can route up or down
  int up_distance = abs(dest_pos - src_pos);
  int down_distance = abs(src_pos + ring_size_ - dest_pos);
  int xpress_cutoff = jump_size_ / 2;
\end{CppCode}
First we compute the distance in the up and down directions.
We also compute the cutoff distance where it is better to jump or step to the next switch.
If going up is a shorter distance, we have

\begin{CppCode}
  if (up_distance <= down_distance) {
    if (up_distance > xpress_cutoff) {
      path.outport = jump_up_port;
      path.dim = UP;
      path.dir = jump;
      path.vc = 0;
    }
    else {
      path.outport = up_port;
      path.dim = UP;
      path.dir = step;
      path.vc = 0;
    }
  }
\end{CppCode}
We then decide if it is better to step or jump.
We do not concern ourselves with virtual channels here and just set it to zero.
Similarly, if the down direction in the ring is better

\begin{CppCode}
  else {
    if (down_distance > xpress_cutoff) {
      path.outport = jump_down_port;
      path.dim = DOWN;
      path.dir = jump;
      path.vc = 0;
    }
    else {
      path.outport = down_port;
      path.dim = DOWN;
      path.dir = step;
      path.vc = 0;
    }
  }
}
\end{CppCode}

For adaptive routing, we need to compute productive paths.
In this case, we only have a single dimension so there is little adaptive routing flexibility.
The only productive paths are the minimal paths.

\begin{CppCode}
void
xpress_ring::get_productive_path(
  int dim,
  const coordinates& src,
  const coordinates& dst,
  routable::path& path) const
{
  minimal_route_to_coords(src, dst, path);
}
\end{CppCode}

We are now ready to use our topology in an application.
In this case, we just demo with the built-in MPI ping all program from \sstmacro.
Here every node in the network sends a point-to-point message to every other node.
There is a parameter file in the \inlineshell{tutorials/programming/toplogy} folder.
To specify the new topology

\begin{ViFile}
# Topology
topology.name = xpress
topology.xpress_ring_size = 10
topology.xpress_jump_size = 5
\end{ViFile}
with application launch parameters

\begin{ViFile}
# Launch parameters
launch_indexing = block
launch_allocation = first_available
launch_app1_cmd = aprun -n10 -N1
launch_app1 = mpi_test_all
\end{ViFile}
The file also includes a basic machine model.

After compiling in the folder, we produce an executable \inlineshell{runsstmac}.
Running the executable, we get the following

\begin{ShellCmd}
Estimated total runtime of           0.00029890 seconds
SST/macro ran for       0.4224 seconds
\end{ShellCmd}
where the \sstmacro wall clock time will vary depending on platform.
We estimate the communication pattern executes and finishes in 0.30 ms.
Suppose we change the jump size to a larger number.
Communication between distant nodes will be faster, but communication between medium-distance nodes will be slower.
We now set \inlinefile{jump_size = 10} and get the output

\begin{ShellCmd}
Estimated total runtime of           0.00023990 seconds
SST/macro ran for       0.4203 seconds
\end{ShellCmd}
We estimate the communication pattern executes and finishes in 0.24 ms, a bit faster.
Thus, this communication pattern favors longer jump links.


