% !TEX root = developer.tex

\chapter{\sstmacro Connectable Interface}\label{chapter:connectables}

\section{Required Functions}

Hardware components communicate via ports. Component 1 sends an event out on one port.
Component 2 receive the event in on another port.
During simulation setup, components must have their ports ``connected'' together.
Creating a connection or link from an output port to an input port requires registering event handlers for each end of the link.
The source component sends events out to a payload handler. 
Upon arriving at the destination component, the payload handler is invoked for that event.
After receiving the event, the destination component can optionally send an ack or credit back to the source.
Thus a link can also have credit handlers registered.

\sstmacro actually provides a thin wrapper around the core SST interface.
In \sstmacro, ports are integers.
In SST core, ports are labeled by strings.
Similarly, payload and credit handlers are not automatically set up in SST core.
\sstmacro forces links and handlers to be set up automatically.

Every hardware component in \sstmacro should inherit from \inlinecode{connectable_component} in \inlinecode{connection.h}
There are four critical abstract functions in the virtual interface. First:

\begin{CppCode}
  virtual void connect_output(
    sprockit::sim_parameters* params,
    int src_outport,
    int dst_inport,
    event_handler* payload_handler) = 0;
\end{CppCode}
This is invoked on the source component of a link giving the port numbers on either end of the link.
It gives the source component the payload handler that will be invoked on the destination component.
The final complication here is the parameters object.
The parameters passed in here are any port-specific parameters.
These include all the default parameters for the port (that may not be port-specific)
plus all parameters in the namespace \inlinecode{portN} for a given port number.
Parameter namespaces are covered in the user's manual.

The next connection function is:
\begin{CppCode}
  virtual void connect_input(
    sprockit::sim_parameters* params,
    int src_outport,
    int dst_inport,
    event_handler* credit_handler) = 0;
\end{CppCode}
Similar to \inlinecode{connect_output}, this is invoked on the destination component of a link.
Instead of giving a payload handler to receive new events,
it receives a credit handler that the destination should send acks and credits to.
The parameters work the same way as the output parameters.

But where do the handlers come from? Connectable objects must implement:

\begin{CppCode}
virtual link_handler* credit_handler(int port) const = 0;

virtual link_handler* payload_handler(int port) const = 0;
\end{CppCode}

These \inlinecode{link_handler} objects are a special instance of \inlinecode{event_handler}.
Each class must return the correct payload and credit handlers for each valid port.
The handler and port will then be passed to the corresponding \inlinecode{connect_output} or \inlinecode{connect_input} function.

\inlinecode{link_handler} objects are created as functors for particular member functions of a class.
They are created through the helper function:

\begin{CppCode}
template <class T, class Fxn>
link_handler* new_link_handler(const T* t, Fxn fxn){
  return new_handler<T,Fxn>(const_cast<T*>(t), fxn);
}
\end{CppCode}
Given a class \inlinecode{Test} with a member function

\begin{CppCode}
void Test::handle_payload(event* ev)
\end{CppCode}
we could create the appropriate \inlinecode{link_handler} as

\begin{CppCode}
link_handler* Test::payload_handler(int port) const {
  return new_link_handler(this, &Test::handle);
}
\end{CppCode}

\section{Example External Component}
An example component source file, corresponding Makefile for generating the external library, and parameter file demonstrating its usage can be found in \inlineshell{skeletons/sst_component_example}.
Some critical things to note from the file \inlineshell{component.cc} are the component registration macro and the Python module generation.
The Python module generation is specific to SST core and is not part of \sstmacro.

The component registration macro is:

\begin{CppCode}
RegisterComponent("dummy", test_component, dummy_switch,
       "test", COMPONENT_CATEGORY_NETWORK,
       "A dummy switch for teaching")
\end{CppCode}
This is similar to the factory registration macro, but extends it for the special case of independent hardware components.
If the component registration macro is used, then the factory registration macro is not required.
The first field is a unique string identifier for the component.
The second field is the parent factory type.
The third field is the actual class name.
The fourth field is the module name matching the Python module name at the top of the \inlinecode{component.cc} file.
The fifth field is a generic component category.
The currently allowed categories are defined in SST core as:

\begin{CppCode}
#define COMPONENT_CATEGORY_UNCATEGORIZED  0x00
#define COMPONENT_CATEGORY_PROCESSOR      0x01
#define COMPONENT_CATEGORY_MEMORY         0x02
#define COMPONENT_CATEGORY_NETWORK        0x04
#define COMPONENT_CATEGORY_SYSTEM         0x08
\end{CppCode}
The final field is a documentation string.

All of the required connection functions are implemented in \inlinecode{component.cc}.

\subsection{Python configuration}
The Python file \inlineshell{run.py} in the same folder shows the simplest possible setup with two components connected by a single both on port 0.
First, we import the necessary modules. The file \inlineshell{component.cc} implements a module called \inlinecode{test} that we load by calling \inlinecode{import sst.test}.
We also load all Python functions provided by the macro library.

\begin{CppCode}
import sst
from sst.macro import *
import sst.test
\end{CppCode}
We then make components, e.g.

\begin{CppCode}
latency="1us"
comp1 = sst.Component("1", "test.dummy_switch")
comp1.addParam("id", 1)
comp1.addParam("latency", latency)
\end{CppCode}
and another component

\begin{CppCode}
comp2 = sst.Component("2", "test.dummy_switch")
comp2.addParam("id", 2)
comp2.addParam("latency", latency)
\end{CppCode}
And finally connect them with a link using a \sstmacro helper function:

\begin{CppCode}
port=0
comp1Id=1
comp2Id=2
makeBiNetworkLink(comp1,comp1Id,port,
                  comp2,comp2Id,port,
                  latency)
\end{CppCode}
The code in the Python script causes \inlinecode{connect_output} and \inlinecode{connect_input} to be invoked on port 0 for each of the components.

\subsection{Makefile}
The Makefile uses compiler wrappers installed with \sstmacro.
These differ from the compiler wrappers used for skeleton applications discussed in the user's manual.

\begin{ViFile}
CXX :=    libsst++
CC :=     libsstcc
CXXFLAGS := -fPIC
\end{ViFile}
All components should be compiled with \inlineshell{-fPIC} for use in shared library.
Making generates a \inlinecode{libtest.so} that can be loaded using the Python setup or through the \inlinefile{external_libs} parameter in a \inlinefile{.ini} file.


