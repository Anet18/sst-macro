% !TEX root = developer.tex

\chapter{\sstmacro Classes}\label{chapter:classes}

\section{Factory Types}\label{sec:factory}
We here introduce factory types, i.e. polymorphic types linked to keywords in the input file.
String parameters are linked to a lookup table, finding a factory that produces the desired type.
In this way, the user can swap in and out C++ classes using just the input file.
There are many distinct factory types relating to the different hardware components.
There are factories for topology, NIC, node, memory, switch, routing algorithm - the list goes on.
Here show how to declare a new factory type and implement various polymorphic instances.
The example files can be found in \inlineshell{tutorials/programming/factories}.

\subsection{Usage}\label{subsec:usage}
Before looking at how to implement factory types, let's look at how they are used.
Here we consider the example of an abstract interface called \inlinecode{actor}.
The code example is found in \inlineshell{main.cc}. The file begins

\begin{CppCode}
#include <sstmac/skeleton.h>
#include "actor.h"

namespace sstmac {
    namespace tutorial {

#define sstmac_app_name rob_reiner

int main(int argc, char **argv)
{
\end{CppCode}
The details of declaring and using external apps is found in the user's manual.
Briefly, \sstmacro (using the \inlinecode{sstmac_app_name} define) reroutes the main function to be callable within a simulation.
From here it should be apparent that we defined a new application with name \inlinecode{rob_reiner}.
Inside the main function, we create an object of type \inlinecode{actor}.

\begin{CppCode}
actor* the_guy = actor_factory::get_param("actor_name", getParams());
the_guy->act();
return 0;
\end{CppCode}
We use the \inlinecode{actor_factory} to create the object.
The value of \inlineshell{actor_name} is read from the input file \inlineshell{parameters.ini} in the directory.
Depending on the value in the input file, a different type will be created.
The input file contains several parameters related to constructing a machine model - ignore these for now.
The important parameters are:

\begin{ViFile}
node {
 app1 {
  name = rob_reiner
  biggest_fan = jeremy_wilke
  actor_name = patinkin
  sword_hand = right
 }
}
\end{ViFile}

Using the Makefile in the directory, if we compile and run the resulting executable we get the output

\begin{ViFile}
Hello. My name is Inigo Montoya. You killed my father. Prepare to die!
Estimated total runtime of           0.00000000 seconds
SST/macro ran for       0.0025 seconds
\end{ViFile}

If we change the parameters:

\begin{ViFile}
node {
 app1 {
  name = rob_reiner
  biggest_fan = jeremy_wilke
  actor_name = guest
  num_fingers = 6
 }
}
\end{ViFile}

we now get the output

\begin{ViFile}
You've been chasing me your entire life only to fail now.
I think that's the worst thing I've ever heard. How marvelous.
Estimated total runtime of           0.00000000 seconds
SST/macro ran for       0.0025 seconds
\end{ViFile}

Changing the values produces a different class type and different behavior.
Thus we can manage polymorphic types by changing the input file.

\subsection{Base Class}\label{subsec:baseClass}
To declare a new factory type, you must include the factory header file

\begin{CppCode}
#include <sprockit/factories/factory.h>

namespace sstmac {
    namespace tutorial {

class actor {
\end{CppCode}


We now define the public interface for the actor class

\begin{CppCode}
 public:
  actor(SST::Params& params);
  
  virtual void act() = 0;

  virtual ~actor(){}
\end{CppCode}
Again, we must have a public, virtual destructor.
Each instance of the actor class must implement the \inlinecode{act} method.

For factory types, each class must take a parameter object in the constructor.
The parent class has a single member variable

\begin{CppCode}
 protected:
  std::string biggest_fan_;
\end{CppCode}

After finishing the class, we need to invoke a macro

\begin{CppCode}
DeclareFactory(actor);
\end{CppCode}
making \sstmacro aware of the new factory type.

Moving to the \inlineshell{actor.cc} file, we see the implementation

\begin{CppCode}
namespace sstmac {
    namespace tutorial {

actor::actor(SST::Params& params)
{
  biggest_fan_ = params->get_param("biggest_fan");
}
\end{CppCode}
We initialize the member variable from the parameter object.  We additionally need a macro

\begin{CppCode}
ImplementFactory(sstmac::tutorial::actor);
\end{CppCode}
that defines certain symbols needed for implementing the new factory type.
For subtle reasons, this must be done in the global namespace.

\subsection{Child Class}\label{subsec:childClass}
Let's now look at a fully implemented, complete actor type.  We declare it

\begin{CppCode}
#include "actor.h"

namespace sstmac {
    namespace tutorial {

class mandy_patinkin :
    public actor
{
 public:
  mandy_patinkin(SST::Params& params);
  
  FactoryRegister("patinkin", actor, mandy_patinkin,
    "He's on one of those shows now... NCIS? CSI?");
\end{CppCode}

We have a single member variable

\begin{CppCode}
 private:
  std::string sword_hand_;
\end{CppCode}

This is a complete type that can be instantiated. 
To create the class we will need the constructor:

\begin{CppCode}
mandy_patinkin(SST::Params& params);
\end{CppCode}

And finally, to satisfy the \inlinecode{actor} public interface, we need

\begin{CppCode}
virtual void act() override;
\end{CppCode}

In the class declaration, we need to invoke the macro \inlinecode{FactoryRegister} to register
the new child class type with the given string identifier.
The first argument is the string descriptor that will be linked to the type.
The second argument is the parent base class. 
The third argument is the specific child type.
Finally, a documentation string should be given with a brief description.
We can now implement the constructor:

\begin{CppCode}
mandy_patinkin::mandy_patinkin(SST::Params& params) :
  actor(params)
{
  sword_hand_ = params->get_param("sword_hand");

  if (sword_hand_ == "left"){
    sprockit::abort("I am not left handed!");
  }
  else if (sword_hand_ != "right"){
      spkt_abort_printf(value_error,
          "Invalid hand specified: %s",
          sword_hand_.c_str());
  }
}
\end{CppCode}
The child class must invoke the parent class method. 
Finally, we specify the acting behavior

\begin{CppCode}
void mandy_patinkin::act()
{
    std::cout << "Hello. My name is Inigo Montoya. You killed my father. Prepare to die!"
              << std::endl;
}
\end{CppCode}

Another example \inlineshell{guest.h} and \inlineshell{guest.cc} in the code folder shows the implementation for the second class.

