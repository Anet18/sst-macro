% !TEX root = developer.tex

\chapter{Statistics Collection}
\label{chapter:stats}

Statistics collection for tracking things like congestion or number of bytes sent is difficult to standardize.
Stats collection must be specifically configured to different components (e.g. NIC, CPU, memory) and types of statistic (histogram, spyplot, timeline).
The stats framework is therefore intended to be highly customizable based on the individual analysis being performed without too many constraints.
There are a few universal features all stats objects must comply with.  
First, any object that collects stats must inherit from \inlinecode{stat_collector} contained in the header \inlinecode{sstmac/common/stats/stat_collector.h}.
This defines a virtual interface that every stats object must comply with.
Second, stats objects should not operate on any global or static data unless absolutely necessary for space constraints.
This means if you have 100K nodes, e.g., each node should maintains its own histogram of message sizes.
While some storage could be saved by aggregating results into a single object,
in many cases the storage overhead is minimal.
This is particularly important for thread safety that stats collection be done on independent, non-interfering objects.
At the very end, the \inlinecode{stat_collector} interface defines hooks for aggregating results if you want, e.g., a global histogram for all nodes.

\section{Setting Up Objects}
We use the example here of a the network interface histogram declared in \inlinecode{nic.h}.

\begin{CppCode}
class nic 
{
  ...
  stat_histogram* hist_msg_size_;
  ...
  nic() : hist_msg_size_(0)
  ...
\end{CppCode}

Here the stats object is initialized to zero.
The \inlinecode{stat_collector} object is a factory type.
Thus individual stat collectors can be associated with string identifiers.
For histogram, we declare in \inlinecode{stat_histogram.cc}

\begin{CppCode}
SpktRegister("histogram", stat_collector, stat_histogram);
\end{CppCode}

Inside the \inlinecode{init_factory_params} function for \inlinecode{nic}, we check if the histogram stats should be activated


\begin{CppCode}
if (params->has_namespace("message_sizes")){
  params->enter_namespace("message_sizes");
  stat_collector* hist = stat_collector_factory::get_optional_param("type", "histogram", params);
  hist_msg_size_ = test_cast(stat_histogram, hist);
  params->exit_namespace("message_sizes");
}
\end{CppCode}

It is possible that you create your own customized histogram object.
However, here we default to the histogram provided by \sstmacro.
The histogram constructor initializes a few parameters internally.

\begin{CppCode}
bin_size_ = params->get_quantity("bin_size");
is_log_ = params->get_optional_bool_param("logarithmic", false);
fileroot_ = params->get_param("fileroot");
\end{CppCode}

defining how large histogram bins are, whether the scale is logarithmic, and finally defining a file root for dumping results later.

This only defines a template histogram object, though.
Recall from earlier sections, almost everything in \sstmacro builds a template object and then clones copies of it to fill out the machine.

\begin{CppCode}
void
nic::clone_into(nic* nic) const
{
  if (hist_msg_size_){
    nic->hist_msg_size_ = hist_msg_size_->clone_me(parent_->id());
  }
\end{CppCode}
If we have built a histogram object, create a copy of it but assign it a unique ID.
This unique ID is important for the runtime to distinguish a stat collector object for node 0 from that for node 1, node 2, etc.

When the simulation begins, every component gets assigned an \inlinecode{event_manager}.
At this time, an object must register all statistics with the runtime

\begin{CppCode}
void
nic::set_event_manager(event_manager* m)
{
  if (hist_msg_size_) m->register_stat(hist_msg_size_);
\end{CppCode}

At this point, the stats object is completely configured.
The stats object copy has been assigned a unique ID and registered with the SST runtime.
Each individual NIC can now start (thread-safely) collecting stats.
Internally in the event manager, all objects with the same file root are grouped together.
Thus the \inlineshell{fileroot} parameter is critical for defining unique groups of stats object.
This is important during simulation post-processing when the event manager wants to aggregate results from each individual node.

\section{Dumping Data}\label{sec:dumping}
The first set of virtual functions that every stats object must provide are

\begin{CppCode}
virtual void
simulation_finished(timestamp end) = 0;

virtual void
dump_local_data() = 0;

virtual void
dump_global_data() = 0;
\end{CppCode}

\inlinecode{simulation_finished} tells the stats object what the final time of the simulation is and allows any final post-processing to be done.
This is particularly useful in time-dependent analyses.  In other cases like message size histograms, it is a no-op.
After the stats object has been notified of the simulation finishing, at some point the event manager will instruct it that it is safe to dump its data.
The next method, \inlinecode{dump_local_data}, dumps the data specific to a given node.
A unique filename based on the ID provided above in the \inlinecode{clone_me} function is created to hold the output.
The last method, \inlinecode{dump_global_data}, dumps aggregate data for all nodes.
Here a unique filename based on the file root parameter is generated.
For the default histogram, a data file and gnuplot script are created.

\section{Reduction and Aggregation}\label{sec:reduceStats}
Before the \inlinecode{dump_global_data} function can be called, an aggregation of results must be performed.
Each stats object is therefore required to provide the functions

\begin{CppCode} 
virtual void
reduce(stat_collector* coll) = 0;

virtual void
global_reduce(parallel_runtime* rt) = 0;
\end{CppCode}
The first function does a local reduce.
The object calling the \inlinecode{reduce} function aggregates data into itself from input parameter \inlinecode{coll}
The event manager automatically loops all objects registered to the same file root and reduces them into a global aggregator.
Once the aggregation is complete across all local copies,
a parallel global aggregation must be performed across MPI ranks.
This can be the most complicated part.
For histograms, this is quite easy.
A histogram is just a vector of integers.
The \sstmacro parallel runtime object provides a set of reduce functions for automatically summing a vector.
For more complicated cases, packing/unpacking of data might need to be performed or more complicated parallel operations.
Once the global reduce is done, the event manager is now safe to call \inlinecode{dump_global_data}.
When developing new stats we recommend running medium-sized jobs as a single thread, multi-threaded, and in MPI parallel to confirm the answer is the same.

For the histogram, the reduce functions are quite simple

\begin{CppCode}
void
stat_histogram::reduce(stat_collector *coll)
{
  stat_histogram* other = safe_cast(stat_histogram, coll);

  /** make sure we have enough bins to hold results */
  int max_num = std::max(counts_.size(), other->counts_.size());
  if (max_num > counts_.size()){
    counts_.resize(max_num);
  }

  /** loop all bins to aggregate results */
  int num_bins = other->counts_.size();
  for (int i=0; i < num_bins; ++i){
    counts_[i] += other->counts_[i];
  }
}
\end{CppCode}

and for the global reduce
\begin{CppCode}
void
stat_histogram::global_reduce(parallel_runtime* rt)
{
  int root = 0;
  /** Align everyone to have the same number of bins */
  int my_num_bins = counts_.size();
  int num_bins = rt->global_max(my_num_bins);
  counts_.resize(num_bins);
  
  /** Now global sum the data vector */
  rt->global_sum(&counts_[0], num_bins, root);
}
\end{CppCode}

\section{Storage Contraints}\label{sec:storageStats}
In some cases, storage constraints prevent each node from having its own copy of the data.
This is particularly important for the fixed-time quanta charts which generate several MB of data even in the reduced, aggregated form.
In this case it is acceptable to operate on global or static data.
However, as much as possible, you should maintain the \emph{illusion} of each component having an individual copy.
For example, a NIC should not declare

\begin{CppCode}
class nic {
 ...
 static ftq_calendar* ftq_;
\end{CppCode}
but instead
\begin{CppCode}
class nic {
 ...
 ftq_calendar* ftq_;
\end{CppCode}

Inside the \inlinecode{ftq_calendar} object you can then declare
\begin{CppCode}
class ftq_calendar {
 ...
 static thread_lock lock_;
 static std::vector<ftq_epoch> results_;
\end{CppCode}
which creates a static, aggregated set of results.
The \inlinecode{ftq_calendar} must ensure thread-safety itself via a thread-lock.

