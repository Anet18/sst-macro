% !TEX root = developer.tex

\chapter{SProCKit}\label{chapter:sprockit}
\sstmacro is largely built on the Sandia Productivity C++ Toolkit (\sprockit),
which is included in the \sstmacro distribution.
Projects developed within the simulator using \sprockit can easily 
move to running the application on real machines while still using the \sprockit infrastructure.
One of the major contributions is reference counted pointer types.
The parameter files and input deck are also part of \sprockit.

\section{Debug}\label{sec:debug}
The goal of the \sprockit debug framework is to be both lightweight and flexible.
The basic problem encountered in \sstmacro development early on was the desire to have very fine-grained control over when and where something prints.
Previously declared debug flags are passed through the \inlinecode{debug_printf} macro.

\begin{CppCode}
debug_printf(sprockit::dbg::mpi,
  "I am MPI rank %d of %d",
  rank, nproc);
\end{CppCode}

The macro checks if the given debug flag is active. 
If so, it executes a \inlinecode{printf} with the given string and arguments.
Debug flags are turned on/off via static calls to

\begin{CppCode}
sprockit::debug::turn_on(sprockit::dbg::mpi);
\end{CppCode}

\sstmacro automatically turns on the appropriate debug flags based on the \inlineshell{-d} command line flag
or the \inlineshell{debug = } parameter in the input file.

Multiple debug flags can be specified via OR statements to activate a print statement through multiple different flags.

\begin{CppCode}
using namespace sprockit;
debug_printf(dbg::mpi | dbg::job_launch,
  "I am MPI rank %d of %d",
  rank, nproc);
\end{CppCode}

Now the print statement is active if either MPI or job launching is going to be debugged.

In \inlineshell{sprockit/debug.h} a set of macros are defined to facilitate the declaration.
To create new debug flags, there are two macros.
The first, \inlinecode{DeclareDebugSlot}, generally goes in the header file to make the flag visible to all files.
The second, \inlinecode{RegisterDebugSlot}, goes in a source file and creates the symbols and linkage for the flag.

\begin{CppCode}
launch.h:
DeclareDebugSlot(job_launch);

launch.cc
RegisterDebugSlot(job_launch);
\end{CppCode}

\section{Serialization}\label{sec:serialize}
Internally, \sstmacro makes heavy use of object serialization/deserialization in order to run in parallel with MPI.
To create a serialization archive, the code is illustrated below. Suppose we have a set of variables

\begin{CppCode}
struct point {
 int x;
 int y;
}
point pt;
pt.x = 0;
pt.y = 2;
int niter = 5;
std::string str = "hello";
\end{CppCode}

We can serialize them to a buffer

\begin{CppCode}
sstmac::serializer ser;
ser.set_mode(sstmac::serializer::PACK);
ser.init(new char[512]);
ser & pt;
ser & niter;
ser & str;
\end{CppCode}
In the current implementation, the buffer must be explicitly given.

To reverse the process for a buffer received over MPI, the code would be

\begin{CppCode}
char* buf = new char[512];
MPI_Recv(buf, ...)
sstmac::serializer;
ser.set_mode(sstmac::serializer::UNPACK);
ser.init(buf);
ser & pt;
ser & niter;
ser & str;
\end{CppCode}

Thus the code for serializing is exactly the same as deserializing. The only change is the mode of the serializer is toggled.

The above code assumes a known buffer size (or buffer of sufficient size).
To serialize unknown sizes, the serializer can also compute the total size first.

\begin{CppCode}
sstmac::serializer ser;
ser.reset();
ser.set_mode(sstmac::serializer::SIZER);
ser & pt;
ser & niter;
ser & str;
int size = ser.sizer.size(); //would be 17 for example above
char* buf = new char[size];
...
\end{CppCode}
The known size can now be used safely in serialization.

The above code only applies to plain-old dataypes and strings.
The serializer also automatically handles STL and Boost containers
through the \inlinecode{<<} syntax.
To serialize custom objects, a C++ class must implement the serializable interface.

\begin{CppCode}
namespace my_ns {
class my_object : 
  public sstmac::serializable,
  public sstmac::serializable_type<my_object>
{
 ImplementSerializable(my_object)
 ...
 void
 serialize_order(sstmac::serializer& ser);
 ...
};
}
\end{CppCode}
The serialization interface requires two inheritances.
The first inheritance from \inlinecode{serializable} is the one ``visible'' to the serializer.
This inheritance forces the object to define a \inlinecode{serialize_order} function.
The second inheritance is used in registering a type descriptor.
The macro \inlinecode{ImplementSerializable} inside the class creates a set of necessary functions.
This is essentially a more efficient RTTI, mapping unique integers to a polymorphic type.
There are cleaner ways of doing this in terms of the interface,
but the current setup is chosen for safety.
The forced inheritance allows more safety checks to ensure types are being set up and used correctly.

In the source file, the type descriptor must be registered.
This is done through the macro

\begin{CppCode}
DeclareSerializable(my_ns::my_object);
\end{CppCode}
This macro should exist in the global namespace.
All that remains now is defining the \inlinecode{serialize_order} in the source file

\begin{CppCode}
void
my_object::serialize_order(sstmac::serializer& ser)
{
  ser & my_int_;
  set << my_double_;
  ...
}
\end{CppCode}

For inheritance, only the top-level parent class needs to inherit from \inlinecode{serializable}.

\begin{CppCode}
class parent_object : 
  public sstmac::serializable
{
...
  void
  serialize_order(sstmac::serializer& set);
...
};

class my_object :
  public parent_object,
  public sstmac::serializable_type<my_object>
{
 ImplementSerializable(my_object)
 ...
 void
 serialize_order(sstmac::serializer& ser);
 ...
};
\end{CppCode}
In the above code, only \inlinecode{my_object} can be serialized.
The \inlinecode{parent_object} is not a full serializable type because no descriptor is registered for it.
Only the child can be serialized and deserialized.
However, the parent class can still contribute variables to the serialization.
In the source file, we would have

\begin{CppCode}
void
my_object::serializer_order(sstmac::serializer& ser)
{
  parent_object::serialize_order(ser);
  ...
}
\end{CppCode}
The child object should always remember to invoke the parent serialization method.

\section{Keyword Registration}\label{sec:keywords}
As stated previously, \sprockit actually implements all the machinery for parameter files.
This is not part of the \sstmacro core.
To avoid annoying bugs, the \sprockit input system requires all allowed values for input parameters to be declared.
This can happen in any source file through static initialization macros.
Only one invocation is allowed per source file,
but keywords can be registered in as many source files as desired.
The macro is used in the global namespace:

\begin{CppCode}
RegisterKeywords("nx", "ny", "nz");
\end{CppCode}
This registers some basic keywords that might be used in a 3D grid application.

In many cases a parameter is an enumerated value or fits a pattern.
\sprockit allows regular expressions to be declared as valid patterns for a keyword.

\begin{CppCode}
StaticKeywordRegisterRegexp my_regexp("particle\\d+");
\end{CppCode}
Here you create a static instance of a keyword registration object.
The constructor registers the regular expression.
Now, any keywords matching the regular expression will be considered valid.


