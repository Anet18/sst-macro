% !TEX root = developer.tex

\newcommand{\nodecls}{\inlinecode{node}\xspace}
\newcommand{\topcls}{\inlinecode{topology}\xspace}
\newcommand{\switchid}{\inlinecode{switch_id}\xspace}
\newcommand{\nodeid}{\inlinecode{node_id}\xspace}

\chapter{Hardware Models}
\label{chapter:hardware}

\section{Connectables}
\label{sec:connectables}
While in common usage, \sstmacro follows a well-defined machine model (see below),
it generally allows any set of components to be connected. 
As discussed in Chapter \ref{chapter:des}, the simulation proceeds by having event components exchange messages,
each scheduled to arrive at a specific time.
\sstmacro provides a generic interface for any set of hardware components to be linked together.
Any hardware component that connects to other components and exchanges messages must inherit from the \inlinecode{connectable} class.
The \inlinecode{connectable} class presents a standard virtual interface

\begin{CppCode}
class connectable
{
 public:
  virtual void
  connect(
    int src_outport,
    int dst_inport,
    connection_type_t ty,
    connectable* mod);

  virtual void
  connect_weighted(
    int src_outport,
    int dst_inport, 
    connection_type_t ty,
    connectable* mod,
    double weight, int red);
};
\end{CppCode}

Any connection is identified by two parameters.
First, a port number must be assigned. 
For example, a switch may have several outgoing connections, each of which must be assigned a unique port number.
Second, a device may make several different \emph{types} of connections.
A network switch might connect to other network switches or it might connect an injection link to a node's network interface.
A network interface might connect to an injector or the memory subsystem.
Given the two parameters, the device is free to implement or store the connection in any way.
Two port numbers must be given.  The output port on the router sending messages must be given, but also the input port on which the destination will receive the packet.  
Both port numbers must be given since the destination may have to return credits to the sender.

A certain style and set of rules is recommended for all connectables.
If these rules are ignored, setting up connections can quicky become confusing and produce difficult to maintain code.

The first and most important rule is that \inlinecode{connectables} never make their own connections.
Some ``meta''-object should create connections between objects.
In general, this work is left to a topology object.
An object should never be responsible for knowing about the ``world'' outside itself.
A topology or interconnect tells the object to make a connection rather than the object deciding to make the connection itself.
This will be illustrated below in \ref{sec:topology}.

The second rule to follow is that a connect function should never call another connect function.
In general, a single call to a connect function should create a single link.
If connect functions start calling other connect functions, you can end up a with a recursive disaster.
If you need a bidirectional link (A $\rightarrow$ B, B $\rightarrow$ A),
two separate function calls should be made

\begin{CppCode}
A->connect(B);
B->connect(A);
\end{CppCode}

rather than having, e.g. A create a bidirectional link.

The first two rules should be considered rigorous. 
A third recommended rule is that all port numbers should be non-negative, and, in most cases, should start numbering from zero.
Negative port numbers are generally used to indicate special meaning, \emph{not} an actual port number.
For example, routers generally return a negative number to indicate a packet has arrived at its destination and should be ejected.

Combining the factory system for polymorphic types and the connectable system for building arbitrary machine links and topologies,
\sstmacro provides flexibility for building essentially any machine model you want.
However, \sstmacro provides a recommended machine structure to guide constructing machine models.

\section{Interconnect}\label{sec:topInterconnect}
For all standard runs, the entire hardware model is driven by the interconnect object.
The interconnect creates nodes, creates network switches, chooses a topology, and connects all of the network endpoints together.
In this regard, the interconnect also choose what types of components are being connected together.
For example, if you were going to introduce some custom FPGA device that connects to the nodes to perform filesystem operations,
the interconnect is responsible for creating it.

To illustrate, here is the code for the interconnect that creates the node objects. 
The interconnect is itself a factory object, configured from a parameter file.

\begin{CppCode}
void
interconnect::init_factory_params(sprockit::sim_parameters* params)
{
  topology_ = topology_factory::get_param("topology_name", params);

  node_tmpl_ = node_factory::get_param("node_name", params, node_id());

  nic_tmpl_ = nic_factory::get_param("nic_name", params, node_tmpl_, this);

  long count = topology_->num_nodes();
  for (long i = 0; i < count; i++) {
    add_node(node_id(i));
  }
}
\end{CppCode}

The interconnect creates a topology and then template objects for the node and NIC.
The topology object tells the interconnect how many nodes to create.

\begin{CppCode}
void
interconnect::add_node(node_id nid)
{
  node* newnode = node_tmpl_->clone(nid);
  nic* newnic = nic_tmpl_->clone(newnode, this);
  newnode->set_nic(newnic);

  nodes_[nid] = newnode;
  available_.insert(nid);
}
\end{CppCode}

Here we use the clone interface, creating copies of the template NIC and node.
The \inlinecode{node_factory} and \inlinecode{nic_factory} 
could have been used to create new nodes from the parameter object instead of cloning.
In general, \sstmacro follows a style of creating a single template from the factory and then cloning it.
This is done for two reasons. 
First, cloning is essentially just a memcpy, which is significantly cheaper than reading parameters and doing string conversion operations.
This can save a lot of time and effort when there are a million hardware objects to create.
Second, when a parameter object arrives in the \inlinecode{init_factory_params} function, it is correctly ``namespaced'' for the local scope.
It may happen that after leaving \inlinecode{init_factory_params}, the interconnect needs to create more nodes and the parameters are no longer available.
The only option at this point is to have a template lying around that can be cloned.

\section{Node}\label{sec:node}
Although the \nodecls can be implemented as a very complex model, it fundamentally only requires a single set of functions to meet the public interface.
The \nodecls must provide \inlinecode{execute_kernel} functions that are invoked by the \inlinecode{operating_system} or other other software objects.
The prototypes for these are:

\begin{CppCode}
  virtual void
  execute_kernel(ami::AMI_COMP_FUNC func, sst_message* data);

  virtual void
  execute_kernel(ami::AMI_COMM_FUNC func, sst_message* data);
\end{CppCode}	

By default, the abstract \nodecls class throws an \inlinecode{sprockit::unimplemented_error}. These functions are not pure virtual.
A node is only required to implement those functions that it needs to do.
The various function parameters are enums for the different operations a node may perform:
computation, communication, or other hardware events.
The distinction between computation and hardware is subtle.
Hardware operations are things like interrupts, device resets, device failures.
They are not necessarily ``kernels'' in the standard parlance.

To illustrate a single example, we show the code that handles the function call

\begin{CppCode}
node->execute_kernel(ami::AMI_COMM_SEND, msg);
\end{CppCode}

leads to

\begin{CppCode}
  switch (func) {
    case sstmac::sw::ami::AMI_COMM_SEND: 
    {
      network_message* netmsg = ptr_safe_cast(network_message, data);
      netmsg->set_fromaddr(my_id_);
      if (netmsg->toaddr() == nodeid_) {
      	/* Intranode send */
      }
      else {
        nic_->send(netmsg);
      }
    }  
\end{CppCode}
All dynamic casts are routed through a special macro \inlinecode{safe_cast} for most types or \inlinecode{ptr_safe_cast} for smart pointer types.

\section{Network Interface (NIC)}\label{sec:nic}
The network interface can implement many services, but the basic public interface requires the NIC to do three things:

\begin{itemize}
\item Inject messages into the network
\item Receive messages ejected from the network
\item Deliver ACKs (acknowledgments) of message delivery
\end{itemize}

For sending messages, the NIC must implement

\begin{CppCode}
  virtual void
  do_send(network_message*payload);
\end{CppCode}
A non-virtual, top-level \inlinecode{send} function performs operations standard to all NICs.
Once these operations are complete, the NIC invokes \inlinecode{do_send} to perform model-specific send operations.
The NIC should only ever send \inlinecode{network_message} types.

For the bare-bones class \inlinecode{null_nic}, the function is

\begin{CppCode}
  injector_->handle(msg);
  if (msg->get_needs_ack()) {
    sst_message* acker = msg->clone_ack();
    schedule(now(), parent_node_, acker);
  }
\end{CppCode}
After injecting, the NIC creates an ACK and delivers the notification to the \nodecls.
In general, all arriving messages or ACKs should be delivered to the node.
The node is responsible for generating any software events in the OS.

For receiving, messages can be moved across the network and delivered in two different ways.

\begin{CppCode}
  virtual void
  recv_chunk(const message_chunk::ptr &chunk,
             network_message*parent_msg);

  virtual void
  recv_whole(network_message*msg);
\end{CppCode}

Depending on the congestion model, a large message (say a 1 MB MPI message) might be broken up into many packets.
These message chunks are moved across the network independently and then reassembled at the receiving end.
Alternatively, for flow models or simple analytical models, the message is not packetized and instead delivered as a single whole.
The methods are not pure virtual.  Depending on the congestion model,  the NIC might only implement chunk receives or whole receives.
Upon receipt, just as for ACKs, the NIC should deliver the message to the node to interpret.
Again, for the bare-bones class \inlinecode{null_nic}, we have

\begin{CppCode}
void
null_nic::recv_chunk(const message_chunk::ptr &chunk,
                     network_message*parent_msg)
{
  bool complete = completion_queue_.recv(chunk);
  if (complete){
    parent_->handle(parent_msg);
  }
}
\end{CppCode}

A special completion queue object tracks chunks and processes out-of-order arrivals,
notifying the NIC when the entire message is done.

\section{Memory Model}\label{sec:memModel}
As with the NIC and node, the memory model class can have a complex implementation under the hood,
but it must funnel things through the a common function.

\begin{CppCode}
  virtual void
  access(sst_message* msg);
\end{CppCode}

The memory model must handle many different types of messages since so many different devices need to access the memory subsystem.
A NIC, the CPU, a GPU, disk may all issue requests to the memory.
In this regard, any message that implements

\begin{CppCode}
  virtual long
  byte_length() const;
\end{CppCode}
should be valid.

\section{Network Switch}\label{sec:networkSwitch}

Unlike the other classes above, a network switch is not required to implement any specific functions.
It is only required to be an \inlinecode{event_handler}, providing the usual \inlinecode{handle(sst_message* msg)}.
The internal details can essentially be arbitrary.
However, the basic scheme for most switches follows the code below for the simplest \inlinecode{packet_switch} model.

\begin{CppCode}
  packet_message::ptr pack = safe_cast(packet_message, msg);
  router_->route(pack);
  int port = pack->rinfo()->port();
  timestamp bw_delay(pack->byte_length() / portbw_[port]);
  timestamp delay = lat_r2r_ + bw_delay;
  event_handler* dest = outports_[port];
  schedule(now()+ delay, dest, msg);
\end{CppCode}
The router object selects the next destination (port).
Then bandwidth and latency terms determine the packet delay.
The packet is then scheduled to the next switch.
Generally, a negative port number indicates ejection.

\section{Topology}
\label{sec:topology}
Of critical importance for the network modeling is the topology of the interconnect.
Common examples are the torus, fat tree, or butterfly.
To understand what these topologies are, there are many resources on the web that do a better job than we could.
Regardless of the actual structure as a torus or tree, the topology should present a common interface to the interconnect and NIC for routing messages.
Here we detail the public interface.
\subsection{Basic Topology}
Not all topologies are ``regular'' like a torus.  Ad hoc computer networks (like the internet) are ordered with IP addresses, but don't follow a regular geometric structure.
The abstract topology base class is intended to cover both cases.
Irregular or arbitrary topology connections are not fully supported yet.

The most important functions in the \topcls class are

\begin{CppCode}
class topology
{

  virtual std::vector<node_id>
  get_nodes_connected_to_switch(switch_id swid) const = 0;

  virtual long
  num_switches() const = 0;

  virtual long
  num_nodes() const = 0;

  virtual void
  connect_objects(connectable_map& objects,
                connectable_factory* cloner) = 0;

  virtual switch_id
  node_to_injector_addr(node_id nodeaddr, int& switch_port) const = 0;

  virtual switch_id
  node_to_ejector_addr(node_id nodeaddr, int& switch_port) const = 0;

  virtual void
  minimal_route_to_switch(
    switch_id current_sw_addr,
    switch_id dest_sw_addr,
    routing_info::path& path) const = 0;

  virtual int
  num_hops_to_node(node_id src, node_id dst) const = 0;


\end{CppCode}

These functions are documented at some length in the \inlineshell{topology.h} header file.
The most important thing to distinguish here are \nodeid and \switchid types.
These are both special opaque integer types that distinguish between a switch or internal object in the topology and a node or network endpoint.
The first few functions just give the number of switches, number of nodes, and finally which nodes are connected to a given switch.
The most critical function for a topology is the \inlinecode{connect_objects} function that takes a list of objects and actually forms the links between them.
Each compute node will be connected to an injector switch and an ejector switch (often the same switch).
The topology must provide a mapping between a node and its ejection and injection points.
Additionally, the topology must indicate a port number or offset for the injection in case the switch has many nodes injecting to it.

Besides just forming the connections, a topology is responsible for routing.
Given a source switch and the final destination, a topology must fill out path information.

\begin{CppCode}
struct path {
  int outport;
  int dim;
  int dir;
  int vc;
}
\end{CppCode}

The most important information is the outport, telling a switch which port to route along to arrive at the destination.
More detail can be given with dimension and direction for some topologies (i.e. +X or -X in a torus).
For congestion models with channel dependencies, the virtual channel must also be given to avoid deadlock.
In general, network switches and other devices should be completely topology-agnostic.
The switch is responsible for modeling congestion within itself - crossbar arbitration, credits, outport multiplexing.
The switch is not able to determine for itself which outport to route along.
The topology tells the switch which port it needs and the switch determines what sort of congestion delay to expect on that port.
This division of labor is complicated a bit by adaptive routing, but remains essentially the same.  More details are given later.

\subsection{Structured Topology}\label{subsec:structuredTopology}

The structured topology assumes a regular, ordered connecting of nodes like a torus.
It is synonymous with any topology that can be mapped onto a coordinate system.
This is true even of tree structures for which the coordinates define which branch/leaf.
The structured topology introduces a few extra virtual functions:

\begin{CppCode}
class structured_topology :
  public topology
{
  ....
  virtual int
  ndimensions() const = 0;	

  virtual void
  compute_switch_coords(switch_id swid, coordinates& coords) const = 0;

  virtual void
  minimal_route_to_coords(
    const coordinates& src_coords,
    const coordinates& dest_coords,
    routing_info::path& path) const = 0;

  virtual int
  minimal_distance(
    const coordinates& src_coords,
    const coordinates& dest_coords) const = 0;

  virtual void
  get_productive_path(
    int dim,
    const coordinates& src,
    const coordinates& dst,
    routing_info::path& path) const = 0;
  ...
};
\end{CppCode}
The structured topology must have a well-defined number of dimensions, such as a 3D-torus or the number of levels in a tree.
The topology must also be able to map any \switchid to a unique set of coordinates.
Given two input sets of input coordinates, it should be able to perform the routing operation or compute the minimal distance between two points.
Finally, for adaptive routing purposes, the topology should map routing requests to the correct path.
For example, if I want to move in the X dimension in a 3D-torus, what path (port) is needed to make a productive step?
It can be that any step along that dimension is unproductive - you've already arrived.
The behavior is then undefined.

\section{Router}\label{sec:router}
The router has a simple public interface

\begin{CppCode}
class router :
  public sprockit::factory_type
{
...
  void
  route(const routable_message::ptr& msg);

  virtual void
  minimal_route_to_node(
    node_id node_addr,
    routing_info::path& path) = 0;
    
  virtual void
  minimal_route_to_switch(
    switch_id sw_addr,
    routing_info::path& path) = 0;
...
};
\end{CppCode}

Here the route function queries the message for algorithm info.

\begin{CppCode}
void
router::route(const routable_message::ptr& msg)
{
  if (!msg->routing_initialized()) {
    init_routing(msg);
  }
  routing_algorithm* algo = algos_[msg->rinfo()->route_algo()];
  if (unlikely(!algo)) {
    spkt_throw_printf(sprockit::value_error,
                     "routing algorithm %s not supported",
                     routing_info::tostr(msg->rinfo()->route_algo()));
  }
  algo->route(msg, this);
}
\end{CppCode}

The actual routing work is done by a routing algorithm object rather than the router.
Different messages may choose different algorithms and we do not want to hard-wire routing to a single algorithm for the entire machine.
The router object exists as a routing metadata store, potentially accumulating congestion or other data over time to inform future routing decisions.
Thus, routing algorithms are static objects with no state - just instructions. Router carry the location-specific state.

For minimal routing on regular routers, the algorithm code is extremely simple

\begin{CppCode}
void
minimal_routing::route(
  const routable_message::ptr& msg,
  router* rter)
{
  rter->minimal_route_to_node(msg->toaddr(), msg->rinfo()->get_path());
}
\end{CppCode}

\begin{CppCode}
void
structured_router::minimal_route_to_node(
  node_id dest_node_addr,
  routing_info::path& path)
{
  switch_id ej_addr = regtop_->node_to_ejector_addr(dest_node_addr, path.dir);
  if (ej_addr == me_) {
    path.dim = topology::EJECT;
    path.vc = 0;
    path.outport = topology::eject_port(path.dir);
  }
  else {
    minimal_route_to_switch(ej_addr, path);
  }
}
...
\end{CppCode}

For adaptive routing, a bit more work is done

\begin{CppCode}
void
minimal_adaptive_routing::route(
  const routable_message::ptr &msg,
  router* rter)
{
  routing_info::path_set paths;
  bool eject  = rter->get_productive_paths_to_node(msg->toaddr(), paths);
  if (eject) {
    msg->rinfo()->assign_path(paths[0]);
    return;
  }
\end{CppCode}
We now query for all possible paths.
Some network switches (not all) provide functionality to query congestion info.

\begin{CppCode}
int test_length = rter->get_switch()->queue_length(paths[i].outport);
\end{CppCode}
from which the lowest congestion path can be chosen.


